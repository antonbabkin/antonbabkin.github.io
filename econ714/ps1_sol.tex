\documentclass{article}
\usepackage{amsmath}
\usepackage{amssymb}
\usepackage{graphicx}
\usepackage{color}
\usepackage{enumerate}
%\usepackage[margin=1in]{geometry}

\newcommand{\R}{\mathbb{R}}
\newcommand{\E}{\mathbb{E}}

\title{Econ 714: Problem Set 1 - Solution}
\author{Anton Babkin}

\begin{document}

{\Large Econ 714: Problem Set 1 - Solution\footnote{By Anton Babkin. \today.}}


\section{}

\begin{enumerate}[(a)]
\item From lecture notes, we have:
  \begin{equation*}
    w_R-z=\frac{\beta p}{1-\beta(1-s)}\int_{w_R}^\infty(w'-w_R)dF(w')
  \end{equation*}

Add and subtract $\frac{\beta p}{1-\beta(1-s)}\left(
\int_0^{w_R}w'dF(w') + w_R\int_0^{w_R}dF(w')\right)$ on the right hand-side. After collecting terms, we get:

  \begin{align*}
    w_R-z&=\frac{\beta p}{1-\beta(1-s)}\left(\int_0^\infty w'dF(w') - w_R \int_0^\infty dF(w') - \int_0^{w_R} (w'-w_R)dF(w')\right)\\
         &=\frac{\beta p}{1-\beta(1-s)}\left(\E w-w_R-\int_0^{w_R} (w'-w_R)dF(w')\right)
  \end{align*}

  where $\E w$ is the expectation of wage under the distribution
  $F(w')$. Multiply by $(1-\beta(1-s))$ both sides and rearrange terms. Finally, by integrating the last integral by parts, we get:

\begin{equation}
(1-\beta(1-s)+\beta p)w_R - (1-\beta(1-s))z = \beta p Ew + \beta p \int_{0}^{w_R}F(w')dw'
\end{equation}

\item If $G$ be a mean preserving spread of $F$, then $\int_{0}^{b}G(w')dw' \geq \int_{0}^{b}F(w')dw'$. Let $h_f(w)=\int_{0}^{w}F(w')dw'$ and  $h_g(w)=\int_{0}^{w}G(w')dw'$. Then, for any $w$, $h_g(w) \geq h_f(w)$ and so reservation wage is (weakly) higher $w_{R,g} \geq w_{R,f}$.\footnote{Let $h(s)=\int_{0}^{s}F(p)dp$, then $h'(s)=F(s)>0$ and $h''(s)=f(s)>0$ so the function $h(s)$ is convex in $s$. See Ljungqvist and Sargent textbook for more detail.}


\item From the lecture notes, we have:

\begin{equation}
w_R - z = \frac{\beta p}{1-\beta(1-s)}\int_{w_R}^{\infty}(w'-w_R)dF(w) \nonumber
\end{equation}

The term $\int_{w_R}^{\infty}(w'-w_R)dF(w)$ is decreasing in $w$. Then, if we re-write this as:

\begin{equation}
\frac{w_R - z}{\int_{w_R}^{\infty}(w'-w_R)dF(w)}=\frac{\beta p}{1-\beta(1-s)} \nonumber
\end{equation}

Then, the left-hand side will be increasing in $w_R$. If $p$ falls, the reservation wage falls as well. 

Steady-state unemployment rate is determined by:

\begin{eqnarray}
up(1-F(w_R)) &=& s(1-u) \nonumber\\
u &=& \frac{{s}}{p(1-F(w_R))+s}
\end{eqnarray}

We cannot really say what will happen to steady-state unemployment rate if $p$ falls without imposing some conditions on $F(w)$.


\end{enumerate}


\section{}

First notice that at the highest level productivity $x=1$ there is no
incentive to search for a new job, because it won't yield a higher
wage. So when agent gets a new job, his value is $W^n(1)$.

Value of the unemployed is almost like in a standard
Mortensen-Pissarides model:
\begin{equation*}
  rU=z+f(W^n(1)-U)
\end{equation*}

Employed worker will not switch his search decision unless a
productivity shock arrives. Once the shock hits, the job is either
destroyed or continues with a new level of productivity - this is when
worker can decide to switch between searching and not.

Value of the non-searching employed:
\begin{equation*}
  rW^n(x)=w(x)+\lambda\left[\int_0^RUdG(x') + \int_R^1\max\{W^n(x'),W^s(x')\}dG(x') - W^n(x)\right]
\end{equation*}\\
\\
\\
(c.) 

Value of the searching employed:
\begin{align*}
  rW^s(x)= & w(x)-\sigma+f\left[W^n(1)-W^s(x)\right] \\
           & +\lambda\left[\int_0^RUdG(x') + \int_R^1\max\{W^n(x'),W^s(x')\}dG(x') - W^s(x)\right]
\end{align*}

\section{}

\begin{enumerate}[(a)]
  \item The social planner's problem is
    \begin{gather*}
      \max_{\{c_t^1,c_t^2\}_{t=0}^\infty}\sum_{t=0}^\infty\beta^t\left[\lambda u(c_t^1)+(1-\lambda)u(c_t^2)\right]\\
      \text{s.t. } c_t^1+c_t^2=e_t^1+e_t^2\quad\forall t
    \end{gather*}
    where we know that $e_t^1=1$ and $e_t^2=0$ for $0\le t<21$, and $e_t^1=0$ and $e_t^2=1$ for $t\ge 21$. Hence the resource constraint can be rewritten as
    \begin{equation}
      \label{eq:res}
      c_t^1+c_t^2=1
    \end{equation}
    Attaching the Lagrangian multipliers $\{\mu_t\}$ to the constraints, the FOC's are
    \begin{align*}
      c_t^1  : &\, \lambda u'(c_t^1)=\mu_t\\
      c_t^2  : &\, (1-\lambda)u'(c_t^2)=\mu_t
    \end{align*}
    Equating the above we have
    \begin{equation}
      \label{eq:foc1}
      \frac{u'(c_t^1)}{u'(c_t^2)}=\frac{1-\lambda}{\lambda}
    \end{equation}
    and using the resource constraint we have
    \begin{equation}
      \label{eq:foc2}
      \frac{u'(c_t^1)}{u'(1-c_t^1)}=\frac{1-\lambda}{\lambda}
    \end{equation}
    Notice that from \eqref{eq:res} we have, for $t\ne t'$, that if
    $c_t^1\ge c_{t'}^2$, then it must be that $c_t^2<c_{t'}^2$. But
    then
    $\frac{u'(c_t^1)}{u'(c_t^2)}<\frac{u'(c_{t'}^1)}{u'(c_{t'}^2)}$,
    which vioates \eqref{eq:foc1}, so it must be that
    \begin{gather*}
      c_t^1=c^1\\
      c_t^2=c^2
    \end{gather*}
    The value of $c_t^1$ (and hence of $c^2$) is given by
    \eqref{eq:foc2} and depends on $\lambda$ and the utility function.
    

  \item A competitive equilibrium is an allocation
    $\{c_t^1,c_t^2\}_{t=0}^\infty$ and a set of prices
    $\{p_t\}_{t=0}^\infty$ such that:
    \begin{itemize}
    \item Agent $i\in\{1,2\}$ maximizes utility:
      \begin{gather}
        \max_{\{c_t^i\}_{t=0}^\infty}\sum_{t=0}^\infty\beta^tu(c_t^i)\nonumber\\
        \text{s.t. }\sum_{t=0}^\infty p_tc_t^i=\sum_{t=0}^\infty p_te_t^i\label{eq:budget}
      \end{gather}
    \item Markets clear:
      \begin{equation}
        \label{eq:mc}
        c_t^1+c_t^2=1
      \end{equation}
    \end{itemize}

  \item Attaching the Lagrangian multiplier $\kappa_i$ to agent $i$'s budget constraint, the FOC of this problem is
    \begin{equation}
      \label{eq:foc3}
      c_t^i :\, \beta^tu'(c_t^i)=\kappa^ip_t
    \end{equation}
    Normalize by setting $p_0=1$, so
    \begin{equation*}
      \kappa^i=u'(c_0^i)
    \end{equation*}
    Combining the FOC's for the two agents we have
    \begin{equation}
      \label{eq:foc4}
      \frac{u'(c_t^1)}{u'(c_t^2)}=\frac{u'(c_0^1)}{u'(c_0^2)}\equiv\eta
    \end{equation}
    Notice that from \eqref{eq:mc} we have that if
    $c_t^1\ge c_{t'}^2$, then it must be that $c_t^2<c_{t'}^2$. But
    then
    $\frac{u'(c_t^1)}{u'(c_t^2)}<\frac{u'(c_{t'}^1)}{u'(c_{t'}^2)}$,
    which vioates \eqref{eq:foc4}, so it must be that
    \begin{equation}
      \label{eq:cons}
      \begin{aligned}
        c_t^1=c^1\\
        c_t^2=c^2
      \end{aligned}
    \end{equation}
    Using this in \eqref{eq:foc3} for agent $i$ for periods $0$ and $t$ we have
    \begin{equation}
      \label{eq:p}
      p_t=\beta^t
    \end{equation}
    Using \eqref{eq:cons} and \eqref{eq:p} in \eqref{eq:budget} we have
      \begin{gather*}
        \sum_{t=0}^\infty \beta^tc^1=\sum_{t=0}^{20} \beta^t
      \end{gather*}
      which implies $c^1=1-\beta^{21}$, and using \eqref{eq:mc} we
      have $c^2=\beta^{21}$.

      Thus, the competitive equilibrium is
      $\{c_t^1,c_t^2\}_{t=0}^\infty$ and $\{p_t\}_{t=0}^\infty$, where
      \begin{align*}
        c_t^1=1-\beta^{21}\quad\forall t\\
        c_t^2=\beta^{21}\quad\forall t\\
        p_t=\beta^t\quad\forall t
      \end{align*}

      Clearly, this competitive equilibrium is Pareto optimal for the
      appropriate choice of $\lambda$ (1st welfare theorem), while the
      Pareto optimum is a competitive equilibrium with transfers (2nd
      welfare theorem).

    \item The price of the claim to consumer 1's endowment process
      must be equal to the price of purchasing an equivalent sequence
      of consumption $c_t^1=1$ for $t=0,1,...,20$ and $c_t^1=0$ for
      $t>20$. The same is true for consumer 2's endowment process and
      for the aggregate endowment process. Thus:
      \begin{gather*}
        p_e^1=\sum_{t=0}^{20}p_t=\sum_{t=0}^{20}\beta^t=\frac{1-\beta^{21}}{1-\beta}\\
        p_e^2=\sum_{t=21}^\infty p_t=\sum_{t=21}^\infty \beta^t=\frac{\beta^{21}}{1-\beta}\\
        p_e^1=p_e^1+p_e^2=\sum_{t=1}^\infty p_t=\frac{1}{1-\beta}\\
      \end{gather*}


\end{enumerate}



\section{}

\begin{enumerate}[(a)]
\item Household problem, $i=1,2$:

  \begin{gather*}
    \max_{\{c_t^i(a^t,s^t)\}_{t=0}^\infty}
    \sum_{t=0}^\infty
    \sum_{(a^t,s^t)}
    \beta^t\log(c_t^i(a^t,s^t))Pr(a^t,s^t)\\
    \text{s.t. } \sum_{t=0}^\infty \sum_{(a^t,s^t)} q_t^0(a^t,s^t)c_t^i(a^t,s^t) = 
    \sum_{t=0}^\infty \sum_{(a^t,s^t)} q_t^0(a^t,s^t)e_t^i(a_t,s_t)\\
    c_t^i(a^t,s^t)\ge 0\\
    e_t^1(a_t,s_t)=a_t+s_t\\
    e_t^2(a^t,s^t)=a_t+1-s_t
  \end{gather*}

\item The Arrow-Debreu competitive equilibrium is the sequence of
  allocations $\{\{c_t^i(a^t,s^t)\}_{t=0}^\infty\}_{i=1}^2$ and prices
  $\{q_t^0(a^t,s^t)\}_{t=0}^\infty$ that solves household problem and
  satisfies market clearing condition:

  \begin{equation*}
    c_t^1(a^t,s^t)+c_t^2(a^t,s^t)=e_t^1(a_t,s_t)+e_t^2(a_t,s_t)
  \end{equation*}

\item Note that with the utility function being $\log(c)$, the Inada
  conditions are satisfied, the solution is interior and budget
  constraint holds with equality, and we can drop the nonnegativity
  constraint. By Negishi algorithm, consider the social planner's
  problem with the Pareto weights $(w^1, w^2)$ and $w^1+w^2=1$:

\begin{gather*}
    \max_{\{\{c_t^i(a^t,s^t)\}_{t=0}^\infty\}_{i=1}^2}
    \sum_{i=1}^2 w^i
    \sum_{t=0}^\infty
    \sum_{(a^t,s^t)}
    \beta^t\log(c_t^i(a^t,s^t))Pr(a^t,s^t)\\
    \text{s.t. } c_t^1(a^t,s^t)+c_t^2(a^t,s^t)=e_t^1(a_t,s_t)+e_t^2(a_t,s_t)
  \end{gather*}

  Attaching the Lagrangian multiplier $\lambda_t$ to the budget
  constraint, FOC w.r.t. $c_t^i(a^t,s^t)$ is:
\begin{equation*}
  \frac{w^i\beta^tPr(a^t,s^t)}{c_t^i(a^t,s^t)}=\lambda_t
\end{equation*}

Divide two conditions for $i=1,2$:
\begin{equation*}
  \frac{w^1}{c_t^1(a^t,s^t)}=  \frac{w^2}{c_t^2(a^t,s^t)}
\end{equation*}

Use this together with
$e_t(a_t,s_t)=e_t^1(a_t,s_t)+e_t^2(a_t,s_t) = 2a_t+1$ and the
feasibility constraint to obtain optimal consumption allocations
$c_t^i(a^t,s^t)=w^ie_t(a_t,s_t)$.

Now turn back to household problem. Attaching the Lagrangian
multiplier $\mu^i$ to the budget constraint of household $i$, obtain
FOC:

\begin{equation*}
  \frac{\beta^tPr(a^t,s^t)}{c_t^i(a^t,s^t)}=\mu^iq_t^0(a^t,s^t)
\end{equation*}

Normalizing the price at date $t=0$ and state $(a_0,s_0)=(0,0)$:
$q_0^0(0,0)=\overline{\pi}_a\overline{\pi}_b$, evaluate the above
expression at state $(a_0,s_0)=(0,0)$, and note that $e_0(0,0)=1$, we
can solve for the Lagrangian multiplier:
\begin{equation*}
  \mu^i=\frac{1}{c_0^i(0,0)}=\frac{1}{w^ie_0(0,0)}=\frac{1}{w^i}
\end{equation*}

Substitute into the first-order condition to obtain the AD securities
price as:
\begin{equation*}
  q_t^0(a^t,s^t)=\frac{\beta^tPr(a^t,s^t)}{\mu^ic_t^i(a^t,s^t)}
  =\frac{w^i\beta^tPr(a^t,s^t)}{w^ie_t(a_t,s_t)}
  =\frac{\beta^tPr(a^t,s^t)}{2a_t+1}
\end{equation*}

Withe price and optimal consumption allocation, substitute into budget
constraint to solve for pareto weight for individual $i=1$ and that
$w^2=1-w^1$:
\begin{align*}
  \sum_{t=0}^\infty \sum_{(a^t,s^t)} \frac{\beta^tPr(a^t,s^t)}{e_t(a_t,s_t)}w^1e_t(a_t,s_t) & =
    \sum_{t=0}^\infty \sum_{(a^t,s^t)} \frac{\beta^tPr(a^t,s^t)}{e_t(a_t,s_t)}e_t^i(a_t,s_t)\\
w^1\sum_{t=0}^\infty \beta^t\sum_{(a^t,s^t)} Pr(a^t,s^t) & =
    \sum_{t=0}^\infty \beta^t\sum_{(a^t,s^t)} Pr(a^t,s^t)\frac{a_t+s_t}{2a_t+1}\\
w^1\sum_{t=0}^\infty \beta^t& =
 \sum_{t=0}^\infty \beta^t\sum_{(a_t,s_t)} Pr(a_t,s_t)\frac{a_t+s_t}{2a_t+1}\\
w^1\sum_{t=0}^\infty \beta^t& =
    \left(\frac{1}{3}\overline{\pi}_a-\frac{1}{3}\overline{\pi}_s-\frac{2}{3}\overline{\pi}_a\overline{\pi}_s+\frac{2}{3}\right)\sum_{t=0}^\infty \beta^t\\
  w^1& =\frac{1}{3}\overline{\pi}_a-\frac{1}{3}\overline{\pi}_s-\frac{2}{3}\overline{\pi}_a\overline{\pi}_s+\frac{2}{3}\\
  w^2&=\frac{1}{3}\overline{\pi}_a-\frac{1}{3}\overline{\pi}_s-\frac{2}{3}\overline{\pi}_a\overline{\pi}_s-\frac{1}{3}
\end{align*}

To understand how $\sum_{(a^t,s^t)} Pr(a^t,s^t)\frac{a_t+s_t}{2a_t+1}$
turns into $\sum_{(a_t,s_t)} Pr(a_t,s_t)\frac{a_t+s_t}{2a_t+1}$ in the
second equation, consider a simpler case with a single two-state
Markov random variable $z_t\in\{0,1\}$. Let stationary distribution be
$(\bar{p}_0,\bar{p}_1)$, and transition probability from state $i$ to
state $j$ be $p_{ij}$.

For $t=0$, $Pr(z_0=0)=\bar{p}_0$ and $Pr(z_0=1)=\bar{p}_1$. At $t=1$,
probability of history $Pr(z^t)=Pr(z_0,z_1)=Pr(z_0)Pr(z_1|z_0)$. For
example, $Pr(0,0)=\bar{p}_0p_{00}$. Now, group summation terms over
histories at $t=1$:
\begin{align*}
  [Pr(0,0)+Pr(1,0)] &+ [Pr(0,1)+Pr(1,1)]\\
  [\bar{p}_0p_{00}+\bar{p}_1p_{10}]&+[\bar{p}_0p_{01}+\bar{p}_1p_{11}]
\end{align*}
But by the definition of stationary distribution,
$\bar{p}_0p_{00}+\bar{p}_1p_{10} = \bar{p}_0$ and
$\bar{p}_0p_{01}+\bar{p}_1p_{11}=\bar{p}_1$, so the sum simplifies to
$\bar{p}_0+\bar{p}_1$. Same reasoning applies $\forall t$.

Now, using the values for $w^i$, can write down AD equilibrium:
\begin{align*}
  c_t^1(a^t,s^t) &= \left(\frac{1}{3}\overline{\pi}_a-\frac{1}{3}\overline{\pi}_s-\frac{2}{3}\overline{\pi}_a\overline{\pi}_s+\frac{2}{3}\right)(2a_t+1)\\
  c_t^2(a^t,s^t) &= \left(\frac{1}{3}\overline{\pi}_a-\frac{1}{3}\overline{\pi}_s-\frac{2}{3}\overline{\pi}_a\overline{\pi}_s-\frac{1}{3}\right)(2a_t+1)\\
q_t^0(a^t,s^t) &=\frac{\beta^tPr(a^t,s^t)}{2a_t+1}
\end{align*}

As always under complete markets, consumption does not depend on
idiosyncratic risk $s_t$. Consumers get a constant fraction of
aggregate endowment which only varies with $a_t$.

\item Price of one-period contingent claims (Arrow securities) can be found as
  \begin{gather*}
    q_{t+1}^t(a^{t+1},s^{t+1})=\frac{q_{t+1}^0(a^{t+1},s^{t+1})}{q_t^0(a^t,s^t)}=\frac{\frac{\beta^{t+1}Pr(a^{t+1},s^{t+1})}{2a_{t+1}+1}}{\frac{\beta^tPr(a^t,s^t)}{2a_t+1}}\\
    =\beta\frac{2a_{t}+1}{2a_{t+1}+1}Pr(a_{t+1},s_{t+1}|a_t,s_t)=\beta\frac{2a_{t}+1}{2a_{t+1}+1}Pr(a_{t+1}|a_t)Pr(s_{t+1}|s_t)
  \end{gather*}

  There are 16 prices in total, for 4 possible states at $t+1$ after
  each of 4 possible states at $t$. With given parameters, prices are:
  \begin{gather*}
    q_{t+1}^t(a_{t+1}=0,s_{t+1}=0|a_t=0,s_t=0)=0.684\\
    q_{t+1}^t(a_{t+1}=0,s_{t+1}=0|a_t=0,s_t=1)=0.152\\
    q_{t+1}^t(a_{t+1}=0,s_{t+1}=0|a_t=1,s_t=0)=0.7695\\
    q_{t+1}^t(a_{t+1}=0,s_{t+1}=0|a_t=1,s_t=1)=0.171\\
    q_{t+1}^t(a_{t+1}=0,s_{t+1}=1|a_t=0,s_t=0)=0.076\\
    q_{t+1}^t(a_{t+1}=0,s_{t+1}=1|a_t=0,s_t=1)=0.608\\
    q_{t+1}^t(a_{t+1}=0,s_{t+1}=1|a_t=1,s_t=0)=0.0855\\
    q_{t+1}^t(a_{t+1}=0,s_{t+1}=1|a_t=1,s_t=1)=0.681\\
    q_{t+1}^t(a_{t+1}=1,s_{t+1}=0|a_t=0,s_t=0)=0.057\\
    q_{t+1}^t(a_{t+1}=1,s_{t+1}=0|a_t=0,s_t=1)=0.0127\\
    q_{t+1}^t(a_{t+1}=1,s_{t+1}=0|a_t=1,s_t=0)=0.5985\\
    q_{t+1}^t(a_{t+1}=1,s_{t+1}=0|a_t=1,s_t=1)=0.133\\
    q_{t+1}^t(a_{t+1}=0,s_{t+1}=0|a_t=0,s_t=0)=0.0063\\
    q_{t+1}^t(a_{t+1}=0,s_{t+1}=0|a_t=0,s_t=1)=0.0507\\
    q_{t+1}^t(a_{t+1}=0,s_{t+1}=0|a_t=1,s_t=0)=0.0665\\
    q_{t+1}^t(a_{t+1}=0,s_{t+1}=0|a_t=1,s_t=1)=0.532\\
  \end{gather*}

\item The one-period ahead riskless claim to one unit of consumption
  depends on the current state of the world, consumer must buy one
  unit of Arrow securities contingent for each possible state of the
  world in the next period. Hence, we have the following prices:
  \begin{itemize}
  \item If $(a_t,s_t)=(0,0)$, the price is $0.684+0.076+0.057+0.0063=0.8233$
  \item If $(a_t,s_t)=(0,1)$, the price is $0.152+0.608+0.0127+0.0507=0.8233$
  \item If $(a_t,s_t)=(1,0)$, the price is $0.7695+0.0855+0.5985+0.0665=1.52$
  \item If $(a_t,s_t)=(1,1)$, the price is $0.171+0.684+0.133+0.532=1.52$
  \end{itemize}

\end{enumerate}



\end{document}
