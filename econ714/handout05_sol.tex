\documentclass{article}
\usepackage{amsmath}
\usepackage{amssymb}
\usepackage{graphicx}
\usepackage{color}
\usepackage{enumerate}
\usepackage{etoolbox}
%\usepackage[margin=1.5in]{geometry}

\newtoggle{sol}
\settoggle{sol}{true} % set false to hide solutions, true to display
\newcommand{\sol}[1]{\iftoggle{sol}{{\color{blue} #1 }}{}}
\newcommand{\R}{\mathbb{R}}
\newcommand{\E}{\mathbb{E}}

\title{Econ 714: Handout 5 \iftoggle{sol}{- Solution}{}}
\author{Anton Babkin}

\begin{document}

\iftoggle{sol}{}{
  \pagenumbering{gobble}
}

{\Large Econ 714: Handout 5 \sol{- Solution}\footnote{By Anton Babkin. This version: \today.}}

\section[]{Optimal taxation with private information\footnote{Adapted
    from Kocherlakota (2004) Wedges and Taxes, AER Papers and
    Proceedings.}}

The model economy has two periods and a unit measure of agents. Each
angent is endowed with $y_1$ units of the single consumption good in
period 1. Consumption can be stored from one period to the next.

In period 2, agents can exert effort to generate consumption. Measure $p_H$ of
the agents are highly skilled, and one unit of their effort generates
$\theta_H$ units of consumption. Measure $p_L$ are low-skilled. For
them, one unit of effort generates $\theta_L$ units of consumption,
where $\theta_L<\theta_H$. $p_H+p_L=1$.

Agents' utility function is
\begin{equation*}
  U(c_1,c_2,l_2)=u(c_1)+\E[u(c_2)-v(l_2)]
\end{equation*}
where $c_t$ is consumption in period $t$ and $l_t$ is effort in period
$t$. $u'>0,u''<0,v'>0,v''>0$.

Timing is as follows. In period 1 agents choose consumption $c_1$ and
savings $S$. In the beginning of period 2 idiosyncratic productivity
$\theta$ is realized, with $Pr(\theta=\theta_H)=p_H$ and
$Pr(\theta=\theta_L)=p_L$. Then the agent chooses $c_2$ and
$l_2$. Productivity and effort are private invormation, but savings
and output $y=\theta l$ are publicly observed.
\begin{enumerate}
\item Characterize first-best allocation. Explain why it is not
  attainable under private information.

\sol{

  When all information is observable, social planner can choose
  allocations conditional on agent's type. Denote by $x^i$ the choice
  of variable $x$ by an agent of realized type $\theta_i$. Note that
  ex ante utility is maximized, and second period variables are chosen
  for every type before realization of uncertainty.

  Social planner's problem:

  \begin{gather*}
    \max_{c_1,S,c_2^i,l_2^i} u(c_1)+\sum_ip_i(u(c_2^i)-v(l_2^i))\\
    \text{s.t. } c_1+S=y_1\\
    p_L c_2^L + p_H c_2^H = S + p_L\theta_Ll_2^L+p_H\theta_Hl_2^H
  \end{gather*}

  Attach Lagrange multiplier $\lambda$ to the combined resource constraint
  (substitute out $S$)
  $c_1+p_L c_2^L + p_H c_2^H = y_1 +
  p_L\theta_Ll_2^L+p_H\theta_Hl_2^H$ and find FOCs:

  \begin{gather*}
    u'(c_1)=\lambda\\
    p_iu'(c_2^i)=p_i\lambda\\
    p_iv'(l_2^i)=p_i\theta_i\lambda
  \end{gather*}

  Summing the second equation over $i$ and equating to the first,
  obtain Euler equation $u'(c_1)=\E(u'(c_2^i))$. Divide second by
  $p_i$ and equate with the first to get $c_1=c_2^L=c_2^H$:
  consumption is equalized across time and across states.

  Use FOC in $l_2^i$ for $i=L,H$ and divide to get
  $\frac{v'(l_2^L)}{v'(l_2^H)}=\frac{\theta_L}{\theta_H}$. Then
  $\theta_L<\theta_H$ implies $l_2^L<l_2^H$: high productivity agents
  put more effort even though their consumption level is the
  same. Nevertheless, it is optimal ex ante, since agents don't know
  what type they will be in period 2.

  With private information, types are not observable to the planner,
  and $\theta_H$ types will claim that they are $\theta_L$ to exert less
  effort. So this first-best allocation is not compatible with
  incentives under private information.

}

\item Formulate social planner's problem with private
  information. Characterize socially optimal allocation: derive the
  Euler equation and show that there is a wedge between intertemporal
  MRS and MRT.

\sol{

  Under private information, allocations can only be conditional on
  agent's reported types. To guarantee that types would be reported
  truthfully, allocations should be such that agents have no
  incentives to pretend that they are a different type. Such
  allocations must satisfy \emph{incentive compatibility}
  constraints. The full social planner's problem becomes:

n  \begin{gather}
    \max_{c_1,S,c_2^i,l_2^i} u(c_1)+\sum_ip_i(u(c_2^i)-v(l_2^i)) \nonumber \\
    \text{s.t. } c_1+p_L c_2^L + p_H c_2^H = y_1 +
    p_L\theta_Ll_2^L+p_H\theta_Hl_2^H \nonumber\\
    u(c_2^H) - v(y_2^H/\theta_H) \ge u(c_2^L)-v(y_2^L/\theta_H) \tag{$IC_H$}\label{eq:ich}\\
    u(c_2^L) - v(y_2^L/\theta_L) \ge u(c_2^H)-v(y_2^H/\theta_L) \tag{$IC_L$}\label{eq:icl}
  \end{gather}


  It can be proved (and we will just assume here) that \eqref{eq:icl} does
  not bind in equilibrium.

  \eqref{eq:ich} always binds in equilibrium. To prove, assume that
  it's not. Then it is possible to increase $c_2^L$ and decrease
  $c_2^H$ so that resource constraint is still
  satisfied. \eqref{eq:icl} still holds, as LHS goes up and RHS goes
  down. And since by assumption \eqref{eq:ich} holds with strict
  inequality, we can always choose sufficiently small change in
  $c_2^H,c_2^L$ so that \eqref{eq:ich} still holds. But such
  alternative allocation is strictly better for aggregate welfare
  because $c_2^L<c_2^H$ and $u'(c_2^L)>u'(c_2^H)$. So we reached a
  contradiction, and hence \eqref{eq:ich} must be binding.

  Rewrite the problem:

  \begin{gather*}
    \max_{c_1,S,c_2^i,l_2^i} u(c_1)+\sum_ip_i(u(c_2^i)-v(l_2^i)) \\
    \text{s.t. } c_1+p_L c_2^L + p_H c_2^H = y_1 +
    p_L\theta_Ll_2^L+p_H\theta_Hl_2^H \\
    u(c_2^H) - v(y_2^H/\theta_H) = u(c_2^L)-v(y_2^L/\theta_H)\\
  \end{gather*}

  Attach multipliers $\lambda$ and $\mu$ to constraints. To derive
  Euler equation, we only need FOCs on consumption.

  \begin{gather*}
    u'(c_1)=\lambda\\
    p_Lu'(c_2^L)-p_L\lambda-\mu u'(c_2^L)=0\\
    p_Hu'(c_2^H)-p_H\lambda+\mu u'(c_2^H)=0\\
  \end{gather*}

  Substitute $\lambda$ from the first equation and solve the last two
  equations for $\mu$, obtain:
  \begin{equation*}
    \frac{p_Lu'(c_2^L)-p_Lu'(c_1)}{u'(c_2^L)}=\frac{p_Hu'(c_1)-p_Hu'(c_2^H)}{u'(c_2^H)}
  \end{equation*}

  Solve for $u'(c_1)$:
  \begin{equation*}
    u'(c_1)=\frac{1}{p_L\frac{1}{u'(c_2^L)}+p_H\frac{1}{u'(c_2^H)}}
  \end{equation*}

  This is the ``reciprocal'' Euler equation:
  $u'(c_1)=\frac{1}{\E[1/u'(c_2^i]}$.

  By Jensen's inequality, $\E[1/u'(c_2^i)]>1/\E u'(c_2^i)$, and from
  Euler equation $u'(c_1)<\E u'(c_2^i)$. In this equilibrium
  $MRS\ne MRT$, because there is an information friction.

}


\item Formulate agent's problem in a decentralized environment, where
  government imposes a tax $\tau(S,y_2)$ in period 2. Assume that
  $p_L=p_H=1/2$. Show that socially optimal marginal tax on savings
  $\tau_S(S,y_2)$ depends on $y_2$ and $\tau_S(S,y_2^H)<\tau_S(S,y_2^L)$.

\sol{

  To decentralize the second-best allocation means to find an
  appropriate tax schedule, such that socially optimal allocation will
  be choosen by utility-maximizing agents.

  Let the tax schedule be $\tau(S,y_2^i)$: it only depends on
  observables (can't condition taxes on $\theta_i$ or $l_2^i$) and can
  be a non-linear function.

  Agent's problem is:

  \begin{gather*}
    \max_{c_1,S,c_2^i,l_2^i} u(c_1)+\sum_ip_i(u(c_2^i)-v(l_2^i))\\
    \text{s.t. } c_1+S=y_1\\
    c_2^i = S + y_2^i - \tau(S,y_2^i)\quad\forall i\\
    y_2^i=\theta_il_2^i
  \end{gather*}

  Substitute budget constraints in the objective function:
  \begin{equation*}
        \max_{S,y_2^i} u(y_1-S)+\sum_ip_i[u(S + y_2^i - \tau(S,y_2^i))-v(y_2^i/\theta_i)]\\
  \end{equation*}

  FOC in $S$ gives Euler equation:
  \begin{equation*}
    u'(c_1)=\sum_ip_iu'(c_2^i)(1-\tau_S(S,y_2^i))
  \end{equation*}

  For agent's solution to coincide with social optimum, allocation
  should also satisfy \eqref{eq:ich} with equality. In other words,
  $\theta_H$-type agent should be indifferent between allocations
  $(c_2^L,y_2^L)$ and $(c_2^H,y_2^H)$. If the agent chooses
  $(c_2^L,y_2^L)$ for any $i$, his EE becomes
  \begin{equation*}
    u'(c_1)=\sum_ip_iu'(c_2^L)(1-\tau_S(S,y_2^L))=u'(c_2^L)(1-\tau_S(S,y_2^L))
  \end{equation*}

  If he chooses different allocations for $i=L,H$, then EE is
  \begin{equation*}
    u'(c_1)=p_Lu'(c_2^L)(1-\tau_S(S,y_2^L))+p_Hu'(c_2^H)(1-\tau_S(S,y_2^H))
  \end{equation*}
  
  Eliminate $u'(c_1)$ and derive:
  \begin{gather*}
    u'(c_2^L)(1-\tau_S(S,y_2^L))=p_Lu'(c_2^L)(1-\tau_S(S,y_2^L))+p_Hu'(c_2^H)(1-\tau_S(S,y_2^H))\\
    (1-p_L)u'(c_2^L)(1-\tau_S(S,y_2^L))=p_Hu'(c_2^H)(1-\tau_S(S,y_2^H))\\
    \frac{1-\tau_S(S,y_2^L)}{1-\tau_S(S,y_2^H)}=\frac{p_H}{1-p_L}\frac{u'(c_2^H)}{u'(c_2^L)}
  \end{gather*}  

  With $p_i=1/2$ and $c_2^L<c_2^H$, RHS $<1$ and
  $\tau_S(S,y_2^L)>\tau_S(S,y_2^H)$. This means that marginal tax on
  savings (wealth) depends on the labor income in the second period
  and is higher for low-productivity agents.

}


\end{enumerate}

\newpage

\section[]{Cash-in-advance\footnote{Spring 2013 problem set.}}

Consider a cash-in-advance model in which there are two types of
goods: $c_1$ requires money $M_t$ to purchase, while $c_2$ can be
purchased on credit. The two goods are technologically equivalent, as
the endowment $e_t$ can be converted one-for-one into either of them,
so $e_t = c_{1t}+c_{2t}$. Suppose that $e_t$ follows a Markov process
with transition density $Q(e'|e)$. A representative agent in this
economy thus solves:
\begin{equation*}
  \max_{\{c_{1t},c_{2t},M_t\}} \E_0\sum_{t=0}^\infty\beta^tu(c_{1t},c_{2t})
\end{equation*}
subject to the budget constraint:
\begin{equation*}
  P_tc_{1t}+P_tc_{2t}=P_te_t+M_t-M_{t+1},
\end{equation*}
and the cash-in-advance constraint:
\begin{equation*}
  P_tc_{1t}\le M_t.
\end{equation*}
\begin{enumerate}
\item Write down the Bellman equation for the representative household
  and find the optimality conditions.

\sol{

Recursive formulation:

\begin{gather*}
  V(M,e)=\max_{c_1,c_2,M'} u(c_1,c_2)+\beta\E[V(M',e')|e]\\
  \text{s.t. } c_1+c_2=e+\frac{M-M'}{P}\\
  Pc_1\le M
\end{gather*}

With Lagrangian multipliers on constraints:
\begin{equation*}
  V(M,e)=\max_{c_1,c_2,M'} u(c_1,c_2)+\eta(e+\frac{M-M'}{P}-c_1-c_2) + \lambda(M-Pc_1) + \beta\E[V(M',e')|e]
\end{equation*}

Taking first order conditions:
\begin{align*}
  [c_1]&: u_1(c_1,c_2)=\eta+\lambda P\\
  [c_2]&: u_2(c_1,c_2)=\eta\\
  [M']&: \beta\E[V_1(M',e')|e]=\eta/P
\end{align*}

Envelope condition:
\begin{equation*}
  [M] : V_1(M,e) = \eta/P + \lambda
\end{equation*}

Combine to obtain Euler equation:
\begin{equation*}
  \frac{\eta}{P}=\beta\left(\frac{\eta'}{P'}+\lambda'\right)
\end{equation*}

If CIA constraint is not binding, $\lambda = 0$:
\begin{equation*}
  u_1(c_1,c_2)=u_2(c_1,c_2), \quad \frac{u_1(c_1,c_2)}{P}=\beta\frac{u_1(c_1',c_2')}{P'}
\end{equation*}

If it is binding, $\lambda>0$:
\begin{equation*}
  u_1(c_1,c_2)>u_2(c_1,c_2), \quad \frac{u_2(c_1,c_2)}{P}=\beta\frac{u_1(c_1',c_2')}{P'}>\beta\frac{u_2(c_1',c_2')}{P'}
\end{equation*}


}

\item Consider a steady state equilibrium in which the endowment is
  constant $e_t=e$, the money supply grows at a constant rate:
  $M_{t+1}=\mu M_t$, and real balances $M_t/P_t$ are constant. What is
  the minimal level of $\mu$ that will support a steady state monetary
  equilibrium? Is such equilibrium Pareto efficient?

\sol{

  Steady state conditions imply $P'=\mu P$, $u_2(c_1,c_2)=const$ and
  $\eta=\eta'$. The Euler equation becomes:
\begin{gather*}
  \mu\eta=\beta(\eta'+\lambda' P')\\
  \mu = \beta + \lambda' \beta P' / \eta
\end{gather*}

Lagrange multiplier $\lambda' \ge 0$, so the minimal level of $\mu$
consistent with household optimization is $\mu = \beta$ with
$\lambda=0$. Such equilibrium is Pareto optimal, since it satisfies
the Friedman rule: real rate of return on money is equal to a
risk-free rate of return: $P/P'=1/\mu=1/\beta$. CIA constraint is
``barely'' binding ($\lambda=0$, but money is held in equilibrium) and
saving in the form of money is as efficient as any other asset.

}

\end{enumerate}

\end{document}
