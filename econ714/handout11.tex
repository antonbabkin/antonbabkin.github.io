\documentclass{article}
\usepackage{amsmath}
\usepackage{amssymb}
\usepackage{graphicx}
\usepackage{color}
\usepackage{enumerate}
\usepackage{etoolbox}
%\usepackage[margin=1.5in]{geometry}

\newtoggle{sol}
\settoggle{sol}{false} % set false to hide solutions, true to display
\newcommand{\sol}[1]{\iftoggle{sol}{{\color{blue} #1 }}{}}
\newcommand{\R}{\mathbb{R}}
\newcommand{\E}{\mathbb{E}}

\title{Econ 714: Handout 11 \iftoggle{sol}{- Solution}{}}
\author{Anton Babkin}

\begin{document}

% \iftoggle{sol}{}{
%   \pagenumbering{gobble}
% }

{\Large Econ 714: Handout 11 \sol{- Solution}\footnote{By Anton Babkin. This version: \today.}}

\section*{Credit constraints and business cycles\footnote{These notes
    present a simplified version of a model from Kocherlakota (2000)
    ``Creating business cycles through credit constraints''.}}

This model presents a mechanism consistent with business cycle
stylized facts without relying on a persistent exogenous TFP
shock. The three business cycle characteristics of interest are:
\begin{enumerate}
\item Size - amplitude of fluctuations is large.
\item Persistence.
\item Asymmetry - downward movements are sharper and quicker than
  upward ones.
\end{enumerate}

Firm's capital $X$ fully depreciates between periods and is used in
production technology $F(X)$, $F'>0, F''<0$. Firm can borrow at
exogenous borrowing rate $r$ up to a limit
$B_{t+1}\le\bar{B}$. Utility is discounted at rate $\beta=1/(1+r)$.

Firm's problem:
\begin{gather*}
  \max_{C,X,B} \sum \beta^t \ln(C_t)\\
  Y_t=F(X_t)\\
  \text{s.t. } C_t+X_{t+1}+B_t(1+r)=Y_t+B_{t+1}\\
  B_{t+1}\le\bar{B}\\
  C_t,X_t\ge 0\\
  X_0,B_0\text{ given}
\end{gather*}

Rewrite in Bellman form with Lagrange multiplier $\lambda_t$:
\begin{equation*}
  V(X_t,B_t)=\max_{X_{t+1},B_{t+1}}\ln(F(X_t)+B_{t+1}-X_{t+1}-B_t(1+r))+\lambda_t(\bar{B}-B_{t+1})+\beta V(X_{t+1},B_{t+1})
\end{equation*}

Derive Euler equations:
\begin{align}
  \text{FOC: } \quad X_{t+1} & : -\frac{1}{C_t}+\beta V_X(t+1)=0\nonumber\\
  B_{t+1} &: \frac{1}{C_t}+\beta V_B(t+1)-\lambda_t=0\nonumber\\
  \text{ENV: } \quad X_t & : V_X(t)=\frac{1}{C_t}F_X(t)\nonumber\\
  B_t & : V_B(t)=-\frac{1}{C_t}(1+r)\nonumber\\
  \text{EE: } \quad & \frac{1}{C_t}=\beta\frac{1}{C_{t+1}}F_X(X_{t+1})\label{eq:eex}\tag{$EE_X$}\\
                            & \frac{1}{C_t}=\beta\frac{1}{C_{t+1}}(1+r)+\lambda_t\label{eq:eeb}\tag{$EE_B$}
\end{align}

We are interested in seeing how GDP $Y_t$ would respond to an
unexpected shock to initial wealth $F(X_0)-B_0(1+r)$.

Let's start by characterizing the steady state equilibrium,
$C_t=C,X_t=X,B_t=B$. It follows from \eqref{eq:eeb} and $\beta(1+r)=1$
that in steady state $\lambda_t=0$, i.e. borrowing constraint is never
binding.

\eqref{eq:eex} becomes $F_X(X_{t+1})=1+r$, so it can be solved for
steady state $X$. Marginal product of capital is equal to the
exogenous rate of return.

Budget constraint in steady state with $B_t=B_0=B$ can be solved for $C$:
\begin{equation*}
C+X+rB=F(X)
\end{equation*}
  
If the economy starts in steady state $X_0=X,B_0=B\le\bar{B}$, it will
stay there forever. Notice that it is possible to have
$B_0=\bar{B}$. If this can still be a steady state equilibrium,
borrowing constraint $B_{t+1}\le\bar{B}$ is ``barely'' binding.

Suppose that the economy starts at steady state and there is an
additive exogenous shock $\Delta$ to the initial wealth
$F(X_0)-B_0(1+r)$.

If the shock is positive, extra resources could be used to increase
capital stock $X_1$, but that would not be optimal because capital is
already at the optimal level. Instead, debt level $B_1$ will be
reduced, so that interest payments $rB_t$ will be smaller, and
consumption $C_t$ higher. No change in $X_t$ implies no change in
$Y_t$: GDP does not respond to a positive wealth shock.

If the shock is negative, but not to big, then debt can be increased
without hitting the credit constraint. Lifetime consumption will be
lower, but again there will be no change in $X_t$ or $Y_t$.

If negative shock is sufficiently large, to accommodate it firms will
have to both increase the debt level to the limit $B_1=\bar{B}$ and to
reduce capital $X_1<X$, implying a drop in output $Y_1$. Subsequent
dynamics is similar to a neoclassical growth model after a negative
shock to capital: output $Y_t$ will gradually converge to the steady
state, accompanied with gradual increase in consumption.

So far, the model has two desirable features: \emph{asymmetry} (only
sufficiently large negative shocks have impact on GDP) and
\emph{persistence} (temporary shock to wealth has long-lasting effect
on output).

Next we will consider \emph{amplification}, i.e. the size of the
output response relative to the initial disturbance. Let's define it
as $A\equiv\frac{Y_1-Y}{\Delta}$. This analysis is easier to perform on
the linearized version of the model. Assume $F(X)=X^\alpha$.

Suppose that the economy starts at the steady state, and
$B_0=\bar{B}$. As the borrowing level is at the limit, a small
negative shock will drive the economy into a downturn. Borrowing will
always be at the limit, and capital, output and consumption will
gradually converge back to the steady state. The dynamic behavior of
the system will be described by the Euler equation \eqref{eq:eex},
budget constraint with constant $B_t=B=\bar{B}$ and initial condition
$X_0$:
\begin{gather*}
  \frac{1}{C_t}=\beta\frac{1}{C_{t+1}}\alpha X_{t+1}^{\alpha-1}\\
  C_t+X_{t+1}+rB=X_t^\alpha+\epsilon_t\\
  X_0=X
\end{gather*}
where $\epsilon_0=\Delta$ and $\epsilon_t=0,t>0$.

Log-linearize around the steady state using lower-case variable names
for percentage deviations, $x_t\approx\frac{X_t-X}{X}$ etc.

\begin{gather*}
  c_t=c_{t+1}+(1-\alpha)x_{t+1},\quad\forall t\\
  \frac{C}{X}c_t+x_{t+1}=\frac{1}{\beta}x_t,\quad t>0\\
  \frac{C}{X}c_0+x_{1}=\frac{1}{\beta}x_0+\frac{\Delta}{X}\\
  x_0=0
\end{gather*}

This is a system of second order linear difference equations with
initial conditions. We can turn it into a single second-order
difference equation by substituting out $c_t$.

\begin{gather}
  x_{t+2}+(-1-\frac{1}{\beta}-\frac{C}{X}(1-\alpha))x_{t+1}+\frac{1}{\beta}x_t=0,\quad t>0\label{eq:de}\\
  x_{2}+(-1-\frac{1}{\beta}-\frac{C}{X}(1-\alpha))x_{1}+\frac{\Delta}{X}=0\label{eq:ini}\\
  x_0=0\nonumber
\end{gather}

\eqref{eq:de} is a homogenous second order difference equation. It's
charactristic equation
$z^2+(-1-\frac{1}{\beta}-\frac{C}{X}(1-\alpha))z+\frac{1}{\beta}=0$ has roots
\begin{equation*}
  z=\frac{m\pm\sqrt{m^2-4/\beta}}{2}
\end{equation*}
where $m=1+\frac{1}{\beta}+\frac{C}{X}(1-\alpha)$.

Roots are real and distinct, in which case the general solution of
equation \eqref{eq:de} is $x_t=\kappa_1 z_1^t+\kappa_2z_2^t$, where
$\kappa_1$ and $\kappa_2$ are constants pinned down by initial
condition.

One can show that $0<z_1<1$ and $z_2>1$. We are only interested in the
stable solution, so we set $\kappa_2=0$. Let $\gamma\equiv
z_1<1$. Then solution is characterized by
$x_{t+1}=\gamma x_t,\quad t>0$. Substitute this back to \eqref{eq:ini}
to solve for $x_1=\beta\gamma\frac{\Delta}{X}$.\footnote{This is shown
  in equation (43) in Kocherlakota (2000), but I could not derive it
  myself.}

Log-linearizing $Y_t=X_t^\alpha$ yields $y_t=\alpha x_t$, and in
steady state $1=\beta\alpha\frac{Y}{X}$, so
\begin{equation*}
  y_1=\alpha\beta\gamma\frac{\Delta}{X}=\gamma\frac{\Delta}{Y}
\end{equation*}

Going back to the definition of amplification ratio:
\begin{equation*}
  A\equiv\frac{Y_1-Y}{\Delta}=\frac{Y_1-Y}{Y}\frac{Y}{\Delta}=y_1\frac{Y}{\Delta}=\gamma
\end{equation*}

So $A=\gamma<1$, i.e. there is a less than one unit change in GDP in
response to a one unit shock to initial wealth. The paper also shows
that if $\bar{B}\to 0$, then $A\to\alpha$ and if $\bar{B}$ gets big
enough, then $A\to 1$. But there is no amplification in a model with
exogenous borowing limit $\bar{B}$.

However in a full version of the model, presented in Kocherlakota
(2000), $A>1$, i.e. there is an amplification of the initial
shock. Check the paper for details, but here is the intuition.

Output is produced using capital and land, $Y=F(X,L)$. $X$ and $L$ are
complementary. Land $L$ is in limited supply and can be traded at
price $Q$. Suppose that the borrowing limit is
$B_{t+1}\le Q_tL_t$. This can be interpreted as collateral constraint:
firms cannot borrow more than the market value of their
non-depreciating assets.

If there is a negative shock to initial wealth, demand for capital
$X^d$ falls. As land is a complementary input, demand for land $L^d$
also falls. Supply of land is fixed, so equilibrium land price $Q$
falls. If the firm was at the limit of it's collateral constraint,
tightening of the constraint means that it has to borrow less. But
this is equivalent to a negative wealth shock, so loop starts over:
$B\Downarrow\to X\Downarrow\to Q\Downarrow\to B\Downarrow\to ...$.


\end{document}




























