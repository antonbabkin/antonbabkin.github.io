\documentclass{article}
\usepackage{amsmath}
\usepackage{amssymb}
\usepackage{graphicx}
\usepackage{color}
\usepackage{enumerate}
%\usepackage[margin=1in]{geometry}

\newcommand{\R}{\mathbb{R}}
\newcommand{\E}{\mathbb{E}}

\title{Econ 714: Final exam - Solution}
\author{Anton Babkin}

\begin{document}

{\Large Econ 714: Final exam - Solution\footnote{By Anton Babkin. \today.}}

\section[]{}
The correct budget constraint for this problem must be, in nominal terms:
\begin{equation*}
  P_tc_t+M_t+\frac{\alpha_{Bt}}{I_t}+P_t\frac{\alpha_{bt}}{R_t}+P_tS_t\alpha_{St}=\tau_{t-1}+M_{t-1}+\alpha_{Bt-1}+P_t\alpha_{bt-1}+P_t(Y_t+S_t)\alpha_{St-1}
\end{equation*}
where $\tau_t = M_t^s-M_t$ is nominal transfer/tax from changing money supply.

At time $t$ decisions are made over $t$-indexed variables, and $t-1$
variables are states resulting from the previous period.

In real terms:
\begin{equation*}
  c_t+m_t+\frac{\alpha_{Bt}}{P_tI_t}+\frac{\alpha_{bt}}{R_t}+S_t\alpha_{St}=\frac{\tau_{t-1}}{P_t}+m_{t-1}\frac{P_{t-1}}{P_t}+\frac{\alpha_{Bt-1}}{P_t}+\alpha_{bt-1}+(Y_t+S_t)\alpha_{St-1}
\end{equation*}

\begin{enumerate}[(a)]
\item With Lagrange multiplier on budget constraint $\lambda_t$, first
  order conditions are:
  \begin{align*}
    [c_t]&:\beta^tu'(c_t)=\lambda_t\\
    [\alpha_{bt}]&:\frac{\lambda_t}{R_t}=\E_t\lambda_{t+1}\\
    [\alpha_{Bt}]&:\frac{\lambda_t}{P_tI_t}=\E_t\lambda_{t+1}\frac{1}{P_{t+1}}\\
    [\alpha_{St}]&:\lambda_tS_t=\E_t\lambda_{t+1}(Y_{t+1}+S_{t+1})\\
    [m_t]&:\beta^tv'(m_t)-\lambda_t+\E_t\lambda_{t+1}\frac{P_t}{P_{t+1}}=0
  \end{align*}

  Substitute out $\lambda_t$ and plug in goods market clearing
  $c_t=Y_t$ to obtain equilibrium pricing conditions:

  \begin{gather*}
    \frac{1}{R_t}=\beta\E_t\frac{u'(Y_{t+1})}{u'(Y_t)}\\
    \frac{1}{I_t}=\beta\E_t\frac{u'(Y_{t+1})}{u'(Y_t)}\frac{P_t}{P_{t+1}}\\
    S_t=\beta\E_t\frac{u'(Y_{t+1})}{u'(Y_t)}(Y_{t+1}+S_{t+1})\\
    1-\frac{v'(m_t)}{u'(Y_t)}=\beta\E_t\frac{u'(Y_{t+1})}{u'(Y_t)}\frac{P_t}{P_{t+1}}=\frac{1}{I_t}
  \end{gather*}

\item Endowment process is
  $\frac{Y_{t+1}}{Y_t}=\exp(\mu+\sigma W_{t+1})$. Using real bond
  pricing equation:
  \begin{align*}
    \frac{1}{R_t}&=\beta\E_t\left(\frac{Y_{t+1}}{Y_t}\right)^{-\gamma}\\
                 &=\beta\E_t\exp(-\gamma\mu-\gamma\sigma W_{t+1})\\
                 &=\beta\exp(-\gamma\mu+\gamma^2\sigma^2/2)\\
    -\log R_t&=\log\beta-\gamma\mu+\gamma^2\sigma^2/2\\
    r_t&=\gamma\mu-\gamma^2\sigma^2/2-\log\beta
  \end{align*}

  Real bonds return positively depends on growth rate $\mu$ and
  negatively on volatility $\sigma$ and patience $\beta$. Effect of
  $\gamma$ is ambiguous.
  
\item
  \begin{align*}
    \pi_t&\equiv\log\E_t\frac{P_{t+1}}{P_{t}}\\
         &=\log\E_t\left(\frac{Y_{t+1}}{Y_t}\right)^a\\
         &=\log\E_t\exp(a\mu+a\sigma W_{t+1})\\
         &=a\mu+a^2\sigma^2/2
  \end{align*}

  If $\sigma=0$, then simply $\pi_t=a\mu$. If $\sigma>0$, this is a
  quadratic equation in $a$ which generally has two roots:
  \begin{equation}\label{eq:a}
    a=\frac{-\mu\pm\sqrt{\mu^2+2\sigma^2\pi_t}}{\sigma^2}
  \end{equation}

\item Using nominal bond pricing equation and solution of class
  $P_t=Y_t^a$:
  \begin{align*}
    \frac{1}{I_t}&=\beta\E_t\left(\frac{Y_{t+1}}{Y_t}\right)^{-\gamma-a}\\
                 &=\beta\exp(-(\gamma+a)\mu+(\gamma+a)^2\sigma^2/2)\\
-\log I_t&=\log\beta-(\gamma+a)\mu+(\gamma+a)^2\sigma^2/2\\
    i_t&=(\gamma+a)\mu-(\gamma+a)^2\sigma^2/2-\log\beta\\
                 &=\gamma\mu-\gamma^2\sigma^2/2-\log\beta
                   +a\mu+a^2\sigma^2/2-\gamma a \sigma^2 - a^2\sigma^2\\
                 &=r_t+\pi_t-\gamma a \sigma^2 - a^2\sigma^2
  \end{align*}
  where $a$ is given by the equation~\eqref{eq:a}.

\item Without risk, $\sigma=0$, Fisher equation holds exactly and in
  unique equilibrium $\pi_t=\bar{i}_t-r_t$, $a=\pi_t/\mu$.

  If $\sigma>0$, there might be two possible inflation levels in
  equilibrium that correspond to the two roots for $a$.

  In these equilibria inflation is a function of the endowment growth,
  so return on the real bond is correlated with
  inflation. Decomposition of nominal interest rate into real interest
  rate and inflation (Fisher equation) now includes an additional
  covariance term - inflation risk - that can take two values for different $a$:

  \begin{align*}
    \frac{1}{I_t}&=\beta\E_t\frac{u'(Y_{t+1})}{u'(Y_t)}\frac{P_t}{P_{t+1}}\\
                 &=\E_t\beta\frac{u'(Y_{t+1})}{u'(Y_t)}\E_t\frac{P_t}{P_{t+1}}+Cov_t\left(\beta\frac{u'(Y_{t+1})}{u'(Y_t)},\E_t\frac{P_t}{P_{t+1}}\right)\\
                 &=\frac{1}{R_t}\E_t\frac{P_t}{P_{t+1}}+Cov_t\left(\beta\left(\frac{Y_{t+1}}{Y_t}\right)^{-\gamma},\left(\frac{Y_{t+1}}{Y_t}\right)^{-a}\right)
  \end{align*}

\end{enumerate}


\section{}
\begin{enumerate}[(a)]
\item
  \begin{gather*}
    rW=w+\lambda(U(s)-W)\\
    rU(s)=z-c(s)+q(s)(W-U(s))
  \end{gather*}

  Solve for $W-U = \frac{w-z+c(s)}{r+\lambda+q(s)}$. Substitute back
  to get steady state values of $U(s)$ and $W$:
\begin{gather*}
  rW = w-\lambda\frac{w-z+c(s)}{r+\lambda+q(s)}\\
  rU(s)=z-c(s)+q(s)\frac{w-z+c(s)}{r+\lambda+q(s)}
\end{gather*}

\item First order condition of the unemployed with respect to $s$:
  \begin{equation}\label{eq:s}
    c'(s)=q'(s)(W-U)=q'(s)\frac{w-z+c(s)}{r+\lambda+q(s)}
  \end{equation}

\item Rewrite \eqref{eq:s} as
  \begin{equation*}
    \frac{c'(s)(r+\lambda)}{q'(s)}+c'(s)s-c(s)=w-z
  \end{equation*}

  Derivative of the LHS with respect to $s$ is
  \begin{align*}
    (r+\lambda)\frac{c''(s)q'(s)-c'(s)q''(s)}{(q'(s))^2}+c''(s)s+c'(s)-c'(s)\\
    =(r+\lambda)\frac{c''(s)q'(s)-c'(s)q''(s)}{(q'(s))^2}+c''(s)s
  \end{align*}
  and is positive since $c''(s)>0$ and $q''(s)<0$.

  So the LHS is increasing in $s$ and the RHS is constant. So $s$
  increases as $w$ increases.

\end{enumerate}

\section{}

\begin{enumerate}[(a)]
\item When price level fluctuates, and not all firms are able to
  adjust, price dispersion results. This causes the relative prices of
  the different goods to vary. If the price level rises, two things
  happen:
  \begin{itemize}
  \item The relative price of firms who have not set their price for a
    while falls, they experience an increase in demand and raise
    output. Firms who have just reset their prices reduce output. This
    production dispersion is inefficient.
  \item Consumers increase consumption of the goods whose relative
    price has fallend a reduce cunsumption of those goods whose
    relative price has risen. This dispersion in consumption reduces
    welfare.
  \end{itemize}

\item Risk premium of a risky return over a risk-free one can be expressed as
  \begin{equation*}
    \frac{E(r_t)-r^f}{\sigma(r)}=\gamma\sigma(\Delta c_t)corr(\Delta c_t,r_t)
  \end{equation*}

  A puzzle is that empirical estimates of this equation imply that
  $\gamma$ needs to be about 27. This is a puzzle, because such high
  levels of risk-aversion imply implausibly high premiums
  invidividuals will be willing to pay to avoid taking lotteries with
  zero expected payoff. Existing micro-level studies suggest that
  $\gamma$ should be in the order of 2 to 3.

  But even if we allow risk-aversion to be very high, it won't resolve
  the puzzle. With CRRA utility $\gamma$ is not only a coefficient of
  relative risk-aversion, but also an inverse of the elastisity of
  intertemporal substitution. For an observed levels of aggregate
  consumption growth, this implies that risk-free rate must be much
  higher than it historically was.

\item Time consistency problem arises when future plans that are
  optimal at a particular point in time become not optimal when that
  future actually comes.

  In Ramsey problem, it is socially optimal to set zero tax on returns
  to capital for all future periods except the initial one, because
  then capital is already in place, and proportional tax effectively
  becomes a lump sum tax and does not distort households'
  incentives. However if the planner is allowed to reoptimize at some
  time in the future, he would choose to deviate from the zero-tax
  plan and tax capital in that period.

\end{enumerate}



\end{document}
