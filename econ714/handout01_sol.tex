\documentclass{article}
\usepackage{amsmath}
\usepackage{amssymb}
\usepackage{graphicx}
\usepackage{color}
\usepackage{etoolbox}

\newtoggle{sol}
\settoggle{sol}{true}
\newcommand{\sol}[1]{\iftoggle{sol}{{\color{blue} #1 }}{}}
\newcommand{\R}{\mathbb{R}}


\title{Econ 714: Handout 1 \iftoggle{sol}{- Solution}{}}
\author{Anton Babkin}

\begin{document}

{\Large Econ 714: Handout 1 \sol{- Solution}\footnote{By Anton Babkin. This version: \today.}}

\section{Mortensen-Pissarides model}

Compared to Pissarides, job destruction rate is endogenous. Each job
has productivity $px$, where $x$ is idiosyncratic. New $x$ arrives at
Poisson rate $\lambda$, drawn from distribution $G$ on
$[0,1]$. Initial draw is $x=1$.

Value of a job is now $J(x)$. If $J(x) \ge 0)$ job kept, if $J(x)<0$
destroyed. Reservation productivity $R$ such that $J(R)=0$.

Job destruction rate: $\lambda G(R)(1-u)$. Job creation:
$m(u,v)=\theta q(\theta)u$, where $\theta=v/u$ is market tightness. Unemployment flow: 
$\dot{u} = \lambda G(R)(1-u) - \theta q(\theta)u$

Steady state (Beveridge curve):

\begin{equation}
  \label{eq:bc}
  \tag{BC}
  u = \frac{\lambda G(R)}{\lambda G(R)+\theta q(\theta)}
\end{equation}

Value functions for the firm:

\begin{equation}
  \label{eq:fv}
  \tag{FV}
  rV=-pc+q(\theta)(J(1)-V)
\end{equation}

\begin{equation}
  \label{eq:fj}
  \tag{FJ}
  rJ(x)=px-w(x)+\lambda\left[\int_R^1J(s)dG(s)-J(x)\right]
\end{equation}

Value functions for the worker:

\begin{equation}
  \label{eq:wu}
  \tag{WU}
  rU=z + \theta q(\theta)(W(1)-U)
\end{equation}

\begin{equation}
  \label{eq:ww}
  \tag{WW}
  rW(x)=w(x)+\lambda\left[\int_R^1W(s)dG(s)+G(R)U-W(x)\right]
\end{equation}

Worker's share of surplus (Nash bargaining):

\begin{equation}
  \label{eq:nb}
  \tag{NB}
  W(x)-U=\beta[W(x)-U+J(x)-V]
\end{equation}

Zero profit: $V=0$.

Exogenous variables: $\lambda,G,m,p,c,z,r,\beta$.

Endogenous variables: $R,\theta,u,v,w,V,J,U,W$.

\subsection{Solving the model}

\begin{enumerate}
\item Wage equation:
  \begin{equation}
    \label{eq:w}
    \tag{w}
    w(x)=z(1-\beta)+\beta p(x+c\theta)
  \end{equation}

\sol{

  From \eqref{eq:fv} and $V=0$:
  \begin{equation*}
    J(1)=\frac{pc}{q(\theta)}
  \end{equation*}

  Substitute into \eqref{eq:nb} with $x=1$:
  \begin{equation*}
    W(1)-U=\frac{\beta}{1-\beta} \frac{pc}{q(\theta)}
  \end{equation*}

  Plug into \eqref{eq:wu}:
  \begin{equation*}
    rU=z+\theta \frac{\beta}{1-\beta} pc
  \end{equation*}

  Multiply \eqref{eq:ww} by $1-\beta$ and subtract \eqref{eq:fj}
  multiplied by $\beta$. Substitute out $W(x)$ and $J(x)$ using
  \eqref{eq:nb} and get:
  \begin{equation*}
    w(x)=\beta px + r(1-\beta)U
  \end{equation*}

  Use previously found expression for $rU$ to derive \eqref{eq:w}.

}

\item Job creation:
  \begin{equation}
    \label{eq:jc}
    \tag{JC}
    (1-\beta)\frac{1-R}{r+\lambda}=\frac{c}{q(\theta)}
  \end{equation}

\sol{

  Plug \eqref{eq:w} into \eqref{eq:fj}:
  \begin{equation}
    \label{eq:wjf}
    (r+\lambda) J(x)=(1-\beta)(px-z)-\beta pc\theta+\lambda\int_R^1J(s)dG(s)
  \end{equation}
  
  Evaluate \eqref{eq:wjf} at $x=R$ and subtract resulting equation
  from \eqref{eq:wjf}, using $J(R)=0$:
  \begin{equation}
    \label{eq:jx}
    (r+\lambda) J(x)=p(1-\beta)(x-R)
  \end{equation}

  Evaluate at $x=1$ using $J(1)=\frac{pc}{q(\theta)}$ and rearrange to
  get \eqref{eq:jc}.

}


\item Job destruction:
  \begin{equation}
    \label{eq:jd}
    \tag{JD}
    \frac{\beta}{1-\beta}c\theta = R-z/p+\frac{\lambda}{r+\lambda}\int_R^1(s-R)dG(s)
  \end{equation}

\sol{

  Use \eqref{eq:jx} to substitute $J(s)$ under integral in \eqref{eq:wjf}:
  \begin{equation*}
    (r+\lambda) J(x)=(1-\beta)(px-z)-\beta pc\theta+\frac{\lambda}{r+\lambda}p(1-\beta)\int_R^1(s-R)dG(s)
  \end{equation*}

  Evaluate at $x=R$ and divide by $p(1-\beta)$ to get \eqref{eq:jd}.

}

\item Solve \eqref{eq:jc} and \eqref{eq:jd} for $R$ and $\theta$, then
  use \eqref{eq:bc} to solve for $u$ and $v$.

\sol{

  We can't derive closed form solutions, but can argue that solution
  is unique since \eqref{eq:jc} is decreasing and \eqref{eq:jd} is
  increasing in $(\theta,R)$ space.

  Graphs can be used to do comparative statics. For example, if $p$
  increases, \eqref{eq:jd} curve shifts to the right, so $R$
  decreases, $\theta$ increases. From \eqref{eq:bc}, $u$ is decreasing
  and from definition of $\theta$, $v$ must increase.

}


\end{enumerate}


\section[]{Problem - McCall model\footnote{August 2012 macro prelim}}

Consider a variation on the basic sequential search model in which there is wage
growth.  Agents are risk neutral and seek to maximize:

\begin{equation*}
  E\sum_{t=0}^\infty\beta^ty_t
\end{equation*}

where $y_t$ is income in period $t$, which comes either from work or
unemployment benefis, and $0< \beta<1$. Suppose that there are no
separations and each unemployed worker is sure to receive an offer
upon searching.  If the wage offer is $w$ in the first period, then
the wage is $w_t = \phi^t w$ after $t$ periods on the job, where $\phi> 1$ and
$\phi\beta< 1$.  The initial wage offer is drawn from a constant
distribution $F(w)$. Unemployed workers earn a constant benefit of $z$.
\begin{enumerate}
\item Write down an unemployed worker's Bellman equation and
  characterize his optimal decision strategy.

\sol{

Start with value of an employed worker with wage $w$:
\begin{equation*}
  W(w) = w + \beta\phi w + \beta^2\phi^2 w + ... = \frac{w}{1-\beta\phi}
\end{equation*}

Value of an unemployed:
\begin{equation*}
  U=z+\beta\int_0^\infty \max\{U,\frac{w}{1-\beta\phi}\}dF(w)
\end{equation*}

Optimal decision is to accept if $w>w_R$ and reject if $w<w_R$, where
at $w_R$ worker is indifferent:
$U=W(w_R)=\frac{w_R}{1-\beta\phi}$. Split integral in two parts:

\begin{equation*}
  \frac{w_R}{1-\beta\phi}=z+\beta\int_0^{w_R} \frac{w_R}{1-\beta\phi}dF(w) +
\beta\int_{w_R}^\infty \frac{w}{1-\beta\phi}dF(w)
\end{equation*}

Add and subtract $\beta\int_{w_R}^\infty\frac{w_R}{1-\beta\phi}dF(w)$ to the RHS:

\begin{equation*}
  \frac{w_R}{1-\beta\phi}=z+\beta \frac{w_R}{1-\beta\phi} +
\frac{\beta}{1-\beta\phi}\int_{w_R}^\infty (w-w_R)dF(w)
\end{equation*}

Rearrange and multiply by $(1-\beta\phi)$:

\begin{equation}
  \label{eq:mc}
  (1-\beta)w_R-z(1-\beta\phi) = \int_{w_R}^\infty (w-w_R)dF(w)
\end{equation}

We can't solve it explicitly, but can characterise solution by
plotting LHS and RHS as functions of $w_R$. LHS is clearly increasing. To show that RHS is decreasing in $w_R$, we need a negative derivative.

We will use the Leibniz's rule:

\begin{equation*}
  \frac{\mathrm{d}}{\mathrm{d}t} \left (\int_{a(t)}^{b(t)} f(x,t)\,\mathrm{d}x \right )= \int_{a(t)}^{b(t)}\frac{ \partial f}{ \partial t}\,\mathrm{d}x \,+\, f\big(b(t),t\big)\cdot b'(t) \,-\, f\big(a(t),t \big)\cdot a'(t)
\end{equation*}

Applying to the RHS yields ($w_R$ plays the role of $t$, $dF(w)\equiv f(w)dw$):

\begin{gather*}
  -\int_{w_R}^\infty dF(w) + \lim_{b\to\infty} [(b-w_R)f(b)\cdot 0] - (w_R - w_R) \cdot 1 =
  -(1-F(w_R)) < 0
\end{gather*}



}


\item Suppose that there are two economies $i = 1, 2$ that differ in
  their wage growth rates, with $\phi_1 > \phi_2$ (both $\phi_i$ still
  satisfy $1 < \phi_i < 1/\beta$). How do the decision strategies
  differ across economies?

\sol{

  From \eqref{eq:mc} it is clear that increase in $\phi$ shifts the
  upward sloping LHS curve up, so solution $w_R$ must be lower,
  i.e. $w_{R1} < w_{R2}$.

  Intuitively, if wage grows faster once employed, it is better to
  start working earlier, so reservation wage is lower.

}

\end{enumerate}

\end{document}

