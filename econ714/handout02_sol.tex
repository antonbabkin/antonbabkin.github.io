\documentclass{article}
\usepackage{amsmath}
\usepackage{amssymb}
\usepackage{graphicx}
\usepackage{color}
\usepackage{etoolbox}
\usepackage[margin=1in]{geometry}

\newtoggle{sol}
\settoggle{sol}{true} % set false to hide solutions, true to display
\newcommand{\sol}[1]{\iftoggle{sol}{{\color{blue} #1 }}{}}
\newcommand{\R}{\mathbb{R}}
\newcommand{\E}{\mathbb{E}}

\title{Econ 714: Handout 2 \iftoggle{sol}{- Solution}{}}
\author{Anton Babkin}

\begin{document}

\iftoggle{sol}{}{
  \pagenumbering{gobble}
}

{\Large Econ 714: Handout 2 \sol{- Solution}\footnote{By Anton Babkin. This version: \today.}}

\section[]{McCall with savings\footnote{Spring 2014 midterm}}

Recall the McCall search model: when a worker is unemployed, he
receives a constant benefit $z$, searches for a job, and receives an
offer $w$ drawn from a distribution $F(w)$. Wage offers are
i.i.d. over time, and if a worker accepts he works forever at that
wage $w$. Now suppose that workers (both employed and unemployed) are
able to borrow or save in an asset $a$ yielding fixed gross return
$R$. The flow budget constraint is thus:
\begin{equation*}
  c_t + a_{t+1} = Ra_t + y_t
\end{equation*}
where here $y_t$ is the worker’s income, which is either $z$ if he is
unemployed or $w$ if he is employed. Assume there is a borrowing
constraint $a_t \ge \underline{a}$ which never binds. Workers seek to maximize:
\begin{equation*}
  \E_0\sum_{t=0}^\infty\beta^tu(c_t)
\end{equation*}
where $0 < \beta < 1$ and $u$ is bounded and satisfies $u'>0$, $u''<0$.

\begin{enumerate}
\item Write down the Bellman equations characterizing the values of
  employed and unemployed workers.
\item Characterize an unemployed worker’s decisions for responding to
  job offers, consumption, and savings as sharply as you can.
\end{enumerate}


\sol {

  Check Kyle Dempsey's solution of this problem for a great and
  detailed explanation (duplicated on my web-page). As you can see
  there, it is enough to derive an Euler equation.

  The only part I'm not quite sure about is determination of
  reservation wage. By timing convention, accept/reject decision is
  made after $a'$ has been choosen, $w_R$ should be a function of
  $a'$. This in turn should be taken into account when taking FOC of
  value function of the unemployed with respect to $a'$. One should be
  careful and use Leibniz's ruls since $a'$ appears both in the
  integrand $U(a'),W(w,a')$ and limits $w_R(a')$ if the integral.

}


\section[]{Lucas tree\footnote{August 2012 macro prelim}}

Suppose that a representative agent has preferences:
\begin{equation*}
  \E_0\sum_{t=0}^\infty\beta^tu(c_t)
\end{equation*}
over the single non-storable consumption good (“fruit”), where
$\gamma > 0$. Her endowment of the good is governed by a Markov
process with transition function $F(x,x')$.
\begin{enumerate}
\item Define a recursive competitive equilibrium with a market in
  claims to the endowment process (“trees”).

\sol{
  Recursive problem of the agent:
  \begin{equation}
    \label{eq:prbl}
    \begin{aligned}
      V(a,x)=&\max_{c,a'}\left(u(c)+\beta\E[V(a',x')|x]\right) \\
      &\text{s.t. } c+pa'=(p+x)a
    \end{aligned}
  \end{equation}

  where $p$ is the market price of a claim/asset $a$.

  Recursive competitive equilibrium is:
  \begin{itemize}
  \item solution of the agent's problem \eqref{eq:prbl}:
    \begin{itemize}
    \item decision rules $c(a,x)$ and $a'(a,x)$,
    \item value function $v(a,x)$;
    \end{itemize}
  \item prices $p(x)$,
  \end{itemize}
such that markets clear:
\begin{itemize}
\item $a'(a,x) = 1$ (assets)
\item $c(a,x) = x$ (goods)
\end{itemize}

}

\item When the consumer has logarithmic utility, $u(c) = log(c)$, what
  is the equilibrium price/dividend ratio of a claim to the entire
  consumption stream? How does it depend on the distribution of
  consumption growth?

\sol{

  Start with characterising solution of the agent's problem with Euler
  equation.

  Substitute $c$ using budget constraint:
  \begin{equation*}
    V(a,x)=\max_{c,a'}\left(u((p+x)a-pa')+\beta\E[V(a',x')|x]\right)
  \end{equation*}

  \begin{align*}
    \text{FOC}(a')&:  pu'(c)=\beta\E[V_1(a',x')|x] \\
    \text{ENV}(a)&: V_1(a,x) = (p+x)u'(c) \\
    \text{EE}&: pu'(c)=\E[\beta u'(c')(p'+x')|x]
  \end{align*}

Rewrite EE with time subscripts:
\begin{equation*}
  p_t = \E_t\beta\frac{u'(c_{t+1})}{u'(c_t)}(p_{t+1}+x_{t+1})
\end{equation*}

Substitute iteratively, using law of iterated expectations,
$\E_t[E_{t+1}[x_{t+2}]]=E_t[x_{t+2}]$:

\begin{align}
  p_t & = \E_t[\sum_{s=1}^T\beta^s\frac{u'(c_{t+s})}{u'(c_t)}x_{t+s} + \beta^T\frac{u'(c_{t+T})}{u'(c_t)}p_{t+T}] \nonumber \\
      & \to \E_t[\sum_{s=1}^\infty\beta^s\frac{u'(c_{t+s})}{u'(c_t)}x_{t+s}]\label{eq:p}
\end{align}

where
$\lim_{T\to\infty}\E_t\beta^T\frac{u'(c_{t+T})}{u'(c_t)}p_{t+T} = 0$
by transversality condition.

Now use market clearing $c_t = x_t$ and log utility $u'(c)=1/c$ to obtain
\begin{equation*}
  p_t = \E_t[\sum_{s=1}^\infty\beta^sx_{t}] = x_t \sum_{s=1}^\infty\beta^s
\end{equation*}

So $\frac{p_t}{x_t} = \frac{\beta}{1-\beta}$, i.e. price to dividend
ratio does not depend on any future information, including
distribution function of dividends.


}

\item Suppose there is news at time $t$ that future consumption will be
  higher. How will prices respond to this news? How does this depend
  on the consumer’s preferences (which could be CRRA, not necessarily
  log)? Interpret your results.

\sol{

  Statement about future consumption implies that there is a certain
  future increase in dividend or it's distribution, since in
  equilibrium $c=x$.
  
  CRRA utility is $u(c) = \frac{c^{1-\gamma}}{1-\gamma}$,
  $u'(c)=c^{-\gamma}$.

  With market clearing, equation~\eqref{eq:p} becomes
  \begin{align*}
    p_t & =\E_t[\sum_{s=1}^\infty\beta^s\frac{x_{t+s}^{-\gamma}}{x_t^{-\gamma}}x_{t+s}] \\
        & =x_t^{\gamma}\E_t[\sum_{s=1}^\infty\beta^sx_{t+s}^{1-\gamma}]
  \end{align*}

  Current price response to future increase in $x$ depends on
  $\gamma$. If $\gamma<1$, i.e. risk aversion is low, then price
  increases. Otherwise, risk aversion dominates dividend growth and
  price falls.

}

\item Suppose that the endowment process is characterized by lognormal
  growth.  That is, $x_{t+l} = x_t \exp(\xi_{t+1})$, where
  $\xi_t \sim N(\mu, \sigma^2)$ i.i.d. What is one-period risk free
  interest rate?  How does it depend on the preference parameter
  $\gamma$?  Interpret your results.

\sol{

  Price of an asset that pays $g(x_{t+1})$ in one period is given by:
  \begin{equation*}
    p^g_t = \E_t[\beta\frac{u'(x_{t+1})}{u'(x_t)}g(x_{t+1})]
  \end{equation*}

  Risk-free discount bond with price $q_t$ pays $g(x_{t+1})=1$. With
  CRRA utility:
  \begin{align*}
    q_t & = \E_t[\beta\frac{x_{t+1}^{-\gamma}}{x_t^{-\gamma}}\times 1] \\
        & = \E_t[\beta\frac{(x_{t}e^{\xi_{t+1}})^{-\gamma}}{x_t^{-\gamma}}] \\
        & = \beta \E_t e^{-\gamma\xi_{t+1}} \\
        & = \beta e^{-\gamma\mu +\gamma^2\sigma^2/2}
  \end{align*}

  Where the last equation is the expectation of log-normal
  distribution.

  So the gross risk-free interest rate is
  \begin{equation*}
    R = 1/q = \frac{1}{\beta} e^{\gamma\mu -\gamma^2\sigma^2/2}
  \end{equation*}

}



\item Consider an option which is bough or sold in period t and
  entitles the current owner to exercise the right to buy one “tree”
  in period t+1 at the fixed price $\bar{p}$ specified at date
  $t$. (The buyer may choose not to exercise this option.) Find a
  formula for the price of this option in terms of the parameters
  describing preferences and endowments.

\sol{

  Call option with strike price $\bar{p}$ pays
  $g(x_{t+1})=\max\{p(x_{t+1})-\bar{p},0\}$.

  So price of such option must be
  \begin{equation*}
    p^g_t = \E_t[\beta\frac{x_{t+1}^{-\gamma}}{x_t^{-\gamma}}\max\{p(x_{t+1})-\bar{p},0\}]
  \end{equation*}

  With a known distribution $F(x,x')$, it can be computed using
  equation~\eqref{eq:p} for $p(x_{t+1})$.

}


\end{enumerate}

\end{document}

