\documentclass{article}
\usepackage{amsmath}
\usepackage{amssymb}
\usepackage{graphicx}
\usepackage{color}
\usepackage{enumerate}
%\usepackage[margin=1in]{geometry}

\newcommand{\R}{\mathbb{R}}
\newcommand{\E}{\mathbb{E}}

\title{Econ 714: Problem Set 3 - Solution}
\author{Anton Babkin}

\begin{document}

{\Large Econ 714: Problem Set 3 - Solution\footnote{By Anton Babkin. \today.}}

\section[]{\footnote{For a more general case see ``Wedges and Taxes''
    by Kocherlakota (2004, AER Papers and Proceedings), and ``Optimal
    Indirect and Capital Taxation'' by Golosov et al. (2003, Review of
    Economic Studies).}}
\begin{enumerate}[(a)]
\item Maximize expected utility subject to resource and incentive compatibility constraints:
  \begin{gather*}
    \max_{c_1(\theta),c_2(\theta),x(\theta)} \E\left[u(c_1) + \beta(u(c_2)-v(l))\right]\\
    \text{s.t. } x=\theta l\\
    \sum_{i}p_i( c_1(\theta_i) + c_2(\theta_i)) = y_1 + \sum_{i}p_ix(\theta_i)\\
    \E\left[u(c_1(\theta_H)) + \beta(u(c_2(\theta_H))-v(x(\theta_H)/\theta_H))\right] \ge
    \E\left[u(c_1(\theta_L)) + \beta(u(c_2(\theta_L))-v(x(\theta_L)/\theta_H))\right]\\
    \E\left[u(c_1(\theta_L)) + \beta(u(c_2(\theta_L))-v(x(\theta_L)/\theta_L))\right] \ge
    \E\left[u(c_1(\theta_H)) + \beta(u(c_2(\theta_H))-v(x(\theta_H)/\theta_L))\right]\\
  \end{gather*}
  where $p_i=1/2$ for $i=L,H$.

\item We need to show that
  $u'(c_1(\theta_i)) = \beta u'(c_2(\theta_i))$.

  Suppose that it's not true and, without loss of generality,
  $u'(c_1(\theta_i)) > \beta u'(c_2(\theta_i))$. Then it is possible
  to find another allocation
  $c_1'(\theta_i)>c_1(\theta_i),c_2'(\theta_i)<c_2(\theta_i)$ that
  satisfies
  $u(c_1'(\theta_i)) + \beta u(c_2'(\theta_i)) = u(c_1(\theta_i)) +
  \beta u(c_2(\theta_i))$
  and $c_1'(\theta_i) + c_2'(\theta_i)
  <c_1(\theta_i)+c_2(\theta_i)$.
  Since new allocation gives same utility, all incentive constraints
  are satisfied. But there are now extra resources available, that can
  be used to increase all types' utilities by the same amount,
  increasing welfare.

  Hence, conjecture $u'(c_1(\theta_i)) > \beta u'(c_2(\theta_i))$ is
  wrong, and it must be that
  $u'(c_1(\theta_i)) = \beta u'(c_2(\theta_i))$.

\item We need to show that
  $\frac{v'(x(\theta_H)/\theta_H)}{u'(c_2(\theta_H))} = \theta_H$.

  Suppose it's not true and, without loss of generality,
  $\frac{v'(x(\theta_H)/\theta_H)}{u'(c_2(\theta_H))} > \theta_H$.
  Consider another allocation $x'(\theta_H)=x(\theta_H)-\Delta_x$,
  $c_2'(\theta_H)=c_2(\theta_H)-\Delta_c$. Choose $\Delta_x$ and
  $\Delta_c$ so that
  $u(c_2'(\theta_H))-v(x'(\theta_H)/\theta_H)=u(c_2(\theta_H))-v(x(\theta_H)/\theta_H)$. For
  that to hold, we need
  $$u'(c_2(\theta_H))\Delta_c=v'(x(\theta_H)/\theta_H)\Delta_x/\theta_H$$
  or
  $$\frac{\Delta_c}{\Delta_x} = \frac{v'(x(\theta_H)/\theta_H)}{\theta_Hu'(c_2(\theta_H))}>1$$
  The last inequality follows from out conjecture. So
  $\Delta_c>\Delta_x$, meaning that resource constraint can be relaxed
  again, allocation is not optimal, hence conjecture is wrong.

  Notice that in this proof reduction in $x(\theta_H)$ increases RHS
  of the incentive constraint for the $\theta_L$ type. But as was
  mentioned in class, this constraint is not binding, hence we can
  always choose a change sufficiently small not to violate it.

\item Suppose that
  $u'(c_1(\alpha)) = \beta \E_{\theta|\alpha}u'(c_2(\theta,\alpha))$
  for a given $\alpha$, i.e. standard Euler equation holds and
  intertemporal allocation is not distorted.

  To simplify notation, omit $\alpha$ and denote
  $c_2(\theta_i,\alpha)$ by $c_2^i$.

  Consider an allocation $c_1' = c_1 + \Delta/u'(c_1)$,
  $c_2'^i = c_2^i - \Delta/u'(c_2^i)$. For an infinitesimal $\Delta$,
  this allocation also satisfies the above Euler equation, i.e. it is
  optimal. Also note that
  $u(c_1') + \beta\E u(c_2'^i) = u(c_1) + \beta \E u(c_2^i))$, so the
  new allocation satisfies all incentive contraints and does not
  change value of the welfare function.

  By Jensen's inequality
  $$\frac{1}{u'(c_1)}=\frac{1}{\E u'(c_2^i)} < \E \frac{1}{u'(c_2^i)}=\sum_i p_i\frac{1}{u'(c_2^i)}$$

  Then
  $$c_1'-c_1 = \frac{\Delta}{u'(c_1)} < \sum_i p_i\frac{\Delta}{u'(c_2^i)}=\sum_ip_i(c_2^i-c_2'^i)$$
  Or
  $$c_1'+\sum_ip_ic_2'^i < c_1+\sum_i p_i c_2^i$$

  Again, new allocation consumes less resources, meaning that welfare
  can be increased, and original allocation can not be optimal. Hence
  our conjecture is wrong and
  $u'(c_1(\alpha)) \ne \beta
  \E_{\theta|\alpha}u'(c_2(\theta,\alpha))$,
  i.e. intertemporal allocation must be distorted.

\end{enumerate}



\section{}

Note: This solution uses the following corrections/clarifications:
\begin{itemize}
\item $E[\epsilon_{t+1}|x_t]=0$,
\item  $g_{t+1}\equiv M_{t+1}/M_t$,
\item $\frac{1}{g_t}\in[\frac{\underline{\epsilon}}{1-\rho},\frac{\bar{\epsilon}}{1-\rho}]$
\item $R(x_{t-1})$ on the RHS of the HH budget constraint.
\end{itemize}

\begin{enumerate}[(a)]
\item $R(x_1)>0$: nominal interest rate on bonds is positive. But
  money yields zero nominal return, so money is dominated by bonds as
  an asset. Hence, it is never optimal to hold excess money and CIA
  constraint must bind.

  Household problem:

  \begin{gather*}
    V(w_t)=\max_{c_t,s_t,B_t,M_t^d} u(c_t)+\beta\E[V(w_{t+1})|x_t]\\
    \text{s.t. } w_t =
    (d_{t-1}\frac{P_{t-1}}{P_t}+p(x_t))s_{t-1}+\frac{B_{t-1}R(x_{t-1})}{P_t}+\frac{M_{t-1}^d-P_{t-1}c_{t-1}}{P_t}\quad\forall t\\
    c_t-\tau_t+p(x_t)s_t+\frac{B_t}{P_t}\le w_t\quad\forall t\\
    P_tc_t\le M_t^d\quad\forall t\\
    c_t\ge 0\quad\forall t\\
  \end{gather*}

  Solution to this functional equation exists and is unique if
  $u(c_t)$ is bounded and continuous, domain of the vector of choice
  variables is convex, and feasibility correspondence $\Gamma(w_t)$
  implied by constraints is nonempty, compact-valued and continuous.

  Value function $V(w_t)$ is strictly increasing if $u(c_t)$ is
  strictly increasing in $w_t$, and $\Gamma(w_t)$ is monotone in
  $w_t$.

  $V(w_t)$ is differentiable if $u(c_t)$ is strictly concave in
  $w_t,c_t,s_t,B_t,M_t^d$ and differentiable in $w_t$ in the interior
  of the feasibility set, and $\Gamma(w_t)$ is convex.

\item Recursive competitive equilibrium:
  \begin{itemize}
  \item Prices $P_t,p(x_t),R(x_{t})$,
  \item Value function $V(w_t)$ and policy functions
    $c_t(w_t),s_t(w_t),B_t(w_t),M_t^d(w_t)$ that solve household
    problem for these prices and tax $\tau_t$,
  \item Markets clear: $c_t=d_t$, $B_t=0$, $M_t^d=M_t$, $s_t=1$,
  \item Government budget holds: $\tau_t=\frac{M_{t+1}-M_t}{P_t}$.
  \end{itemize}

\item $c_t\ge 0$ never binds if we assume $u'(0)=\infty$, and budget
  constraint always binds if $u'(c_t)>0$.

  CIA constraint $P_tc_t\le M_t^d$ always binds if we assume
  $R(x_t)>1$, so we can substitute out $M_t^d$.

  Rewrite household problem as:
  \begin{gather*}
    V(w_t)=\max_{c_t,s_t,B_t} u(c_t)+\beta\E[V(w_{t+1})|x_t]\\
    \text{s.t. } w_t =
    (d_{t-1}\frac{P_{t-1}}{P_t}+p(x_t))s_{t-1}+\frac{B_{t-1}R(x_{t-1})}{P_t}+\frac{M_{t-1}^d-P_{t-1}c_{t-1}}{P_t}\\
    c_t=w_t+\tau_t-p(x_t)s_t-\frac{B_t}{P_t}
  \end{gather*}

  Plug in $c_t$ and $w_{t+1}$, and find optimality conditions:
  \begin{align*}
    FOC[s_t]&:u'(c_t)p(x_t)=\beta\E V'(w_{t+1})(d_{t}\frac{P_{t}}{P_{t+1}}+p(x_{t+1}))\\
    FOC[B_t]&:\frac{u'(c_t)}{P_t}=\beta\E V'(w_{t+1})\frac{R(x_{t})}{P_{t+1}}\\
    ENV[w_t]&:V'(w_t)=u'(c_t)
  \end{align*}

  These imply two Euler equations that can be used to price bonds and
  Lucas tree:

  \begin{gather}
    u'(c_t)p(x_t)=\beta\E u'(c_{t+1})(d_{t}\frac{P_{t}}{P_{t+1}}+p(x_{t+1}))\label{eq:ees}\tag{$EE_s$}\\
    \frac{u'(c_t)}{P_t}=\beta\E u'(c_{t+1})\frac{R(x_{t})}{P_{t+1}}\label{eq:eeb}\tag{$EE_B$}
  \end{gather}

\item We will guess and verify that $R(x_t)>1$.

  As was discussed above, under this guess CIA constraint binds. Use
  $u'(c_t)=1/c_t$, $c_t=d_t$ and $P_tc_t=M_t$ in
  equation~\eqref{eq:eeb}:\
  \begin{align*}
    \frac{1}{c_tP_t}&=\beta\E\frac{R(x_t)}{c_{t+1}P_{t+1}}\\
    \frac{1}{R(x_t)}&=\beta\E\frac{c_tP_t}{c_{t+1}P_{t+1}}\\
                    &=\beta\E\frac{M_t}{M_{t+1}}\\
                    &=\beta\E[\rho\frac{1}{g_t}+\epsilon_{t+1}]\\
                    &=\beta\frac{\rho}{g_t}\\
    R(x_t)&=g_t\frac{1}{\beta\rho}
  \end{align*}

  Under assumptions on the stochastic process for $g_t$,
  $\frac{1}{g_t}\le\frac{\bar{\epsilon}}{1-\rho}<\frac{1}{\beta\rho}$,
  or $g_t>\beta\rho$. It follows that $R(x_t)>1$, guess verified.

\item Recursively substitute equation~\eqref{eq:ees}:
  \begin{align*}
    \frac{p(x_t)}{c_t}&=\beta\E\frac{d_tP_t}{c_{t+1}P_{t+1}} + \beta\E\frac{p(x_{t+1})}{c_{t+1}}\\
    p(x_t)&=\beta\E d_t\frac{c_tP_t}{c_{t+1}P_{t+1}} + \beta\E\frac{c_t}{c_{t+1}}p(x_{t+1})\\
                      &=\beta\E d_t\frac{M_t}{M_{t+1}}+\beta^2\E d_{t+1}\frac{M_{t+1}}{M_{t+2}}\frac{c_t}{c_{t+1}}+\beta^2\E\frac{c_t}{c_{t+2}}p_{t+2}\\
                      &=...\\
                      &=d_t\sum_{i=1}^\infty\beta^i\E\frac{M_{t+i-1}}{M_{t+i}}\\
                      &=d_t\sum_{i=1}^\infty\beta^i\rho^i\frac{1}{g_t}\\
                      &=d_t\frac{\beta\rho}{1-\beta\rho}\frac{M_{t-1}}{M_t}
  \end{align*}

\item From CIA constraints:
  \begin{equation*}
    \frac{P_{t+1}}{P_t}=\frac{M_{t+1}}{M_t}\frac{d_{t+1}}{d_t}
  \end{equation*}
  And 
  \begin{equation*}
    p(x_t)=d_t\frac{\beta\rho}{1-\beta\rho}\frac{M_{t-1}}{M_t}
  \end{equation*}
  Clearly, for a given $\rho$ increase in money growth rate
  $\frac{M_{t+1}}{M_t}$ increases inflation and decreases price of
  Lucas trees.


\end{enumerate}


\section[]{\footnote{June 2013 Macro Prelim Exam. Solution by Kyle Dempsey}}


\subsubsection*{a. HJB equations}

The value functions of agents who own a house, $V_{1}$, and those
who don't, $V_{0}$, are given by
\begin{eqnarray}
rV_{1} & = & u+\alpha(1-H)\pi_{0}\pi_{1}[p+V_{0}-V_{1}]\\
rV_{0} & = & \alpha H\pi_{0}\pi_{1}[-p+V_{1}-V_{0}].
\end{eqnarray}
Given these two value functions, we can subtract the second from the
first in order to find the welfare gain, $\Delta\equiv V_{1}-V_{0}$:
\begin{equation}
\Delta=\frac{u+\alpha\pi_{0}\pi_{1}p}{r+\alpha\pi_{0}\pi_{1}}.\label{eq:delta}
\end{equation}



\subsubsection*{b. Optimal buying and selling strategies}

In order to characterize each type of agents buying and selling strategies,
we just compare the value of the action to its alternative, 0. For
sellers, we have
\begin{equation}
\pi_{1}=\begin{cases}
1 & \mbox{if }p-\Delta\ge0\\{}
[0,1] & \mbox{if }p-\Delta=0\\
0 & \mbox{if }p-\Delta\le0
\end{cases}
\end{equation}
Given equation (\ref{eq:delta}), we can evaluate the conditioning
expression:
\begin{equation}
p-\Delta=\frac{(r+\alpha\pi_{0}\pi_{1})p-u-\alpha\pi_{o}\pi_{1}p}{r+\alpha\pi_{0}\pi_{1}}=\frac{rp-u}{r+\alpha\pi_{0}\pi_{1}}=0\iff p=\frac{u}{r}\label{eq:surplus}
\end{equation}
The two equations above pin down the optimal selling strategy for
agents with houses.

Turning to the potential buyers of houses, we can proceed in a similar
fashion:
\begin{equation}
\pi_{0}=\begin{cases}
1 & \mbox{if }\Delta-p\ge0\\{}
[0,1] & \mbox{if }\Delta-p=0\\
0 & \mbox{if }\Delta-p\le0
\end{cases}
\end{equation}
This equation, together with (\ref{eq:surplus}), determines the optimal
strategy for agents without houses. Note that under the assumption
that agents trade when they are indifferent, we can set $\pi_{0}=\pi_{1}=1$
when $\Delta=p$.


\subsubsection*{c. Equilibrium price and effects of housing supply}

In order to determine the equilibrium house price $p$, we can consider
equilibria in which houses trade, i.e. ones in which $\pi\equiv\pi_{0}\pi_{1}>0$.
Given the equations for $\pi_{0}$ and $\pi_{1}$ in part b above,
this is clearly only possible when $\Delta=p$, and so the equilibrium
price is given by:
\begin{equation}
p=\frac{u}{r}
\end{equation}


Observe that the price of houses $p$ does not depend on housing supply
$H$! This result makes sense because the price is determined in bilateral
meetings, where regardless of the supply of houses in the overall
economy one agent has a house and another does not. That is, \textbf{conditional
on meeting}, the aggregate housing stock does not matter.





\end{document}
