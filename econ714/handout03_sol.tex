\documentclass{article}
\usepackage{amsmath}
\usepackage{amssymb}
\usepackage{graphicx}
\usepackage{color}
\usepackage{enumerate}
\usepackage{etoolbox}
%\usepackage[margin=1in]{geometry}

\newtoggle{sol}
\settoggle{sol}{true} % set false to hide solutions, true to display
\newcommand{\sol}[1]{\iftoggle{sol}{{\color{blue} #1 }}{}}
\newcommand{\R}{\mathbb{R}}
\newcommand{\E}{\mathbb{E}}

\title{Econ 714: Handout 3 \iftoggle{sol}{- Solution}{}}
\author{Anton Babkin}

\begin{document}

\iftoggle{sol}{}{
  \pagenumbering{gobble}
}

{\Large Econ 714: Handout 3 \sol{- Solution}\footnote{By Anton Babkin. This version: \today.}}

\section{Term structure of interest rates}

Consider the following environment. The economy has a Lucas tree that
issues an uncertain dividend $s_{t}$ each period, which follows the
Markov transition function $Q(s,s')$. Define the risk-free (gross)
interest rate between periods $t$ and $t+j$ as $R_{jt}$ (corresponding
price $q_{jt}=1/R_{jt}$), measured in units of the $t+j$ consumption
good per the time $t$ consumption good. Assume that all bonds are
zero-coupon. Let $b_{jt}$ denote the holdings of the bond with
maturity $j$ issued at time $t$, and let $a_{t}$ denote holdings of the
tree. Suppose the representative agent has time-additive utility such
that $u'>0,u''<0,$ and $\lim_{c\to0}u'(c)=\infty$.
\begin{enumerate}
\item Assume that the only bonds in the market are of 1- and 2-period maturity,
and each is in zero net supply. Write down the agent's Bellman equation
and the relevant budget constraint and market clearing conditions
(i.e. define an equilibrium).
\item Solve for the prices of all three assets. Does the relationship 
\[
q_{2t}=q_{1t}\mathbb{E}_{t}(q_{1t+1})
\]
hold? Why or why not? If not, what changes can we make to the model
so that it does?
\item The \textbf{term structure} is the collection of yields to maturity
across dates of maturity for all bonds within a certain class. The
\textbf{yield to maturity }of a bond is the return earned by an investor
who purchases the bond today and holds it to maturity, collecting
all coupon payments along the way and the return of principal at the
end. Assume that the dividend process is i.i.d. to ease the computation.
Compute the yields to maturity and therefore term structure for the
risk-free assets in this economy. Under what conditions is the yield
curve increasing? 
\item BONUS (an easy one): Generalize your answer to part 3 to the case
when the economy has bonds with maturities 1,2,...$j$.\end{enumerate}


\sol{

This question is adapted from Section 13.8 of LS.

\begin{enumerate}
\item The budget constraint for the representative agent given this environment
is
\begin{equation}
\label{eq:tsbud}
c_{t}+p_{t}a_{t+1}+q_{1t}b_{1t}+q_{2t}b_{2t}\le(p_{t}+s_{t})a_{t}+b_{1,t-1}+q_{1t}b_{2,t-1}
\end{equation}
Notice that on the LHS $b_{1t}$ is indistinguishable from $b_{2,t-1}$,
so we can ignore the latter. Similarly, on the RHS $b_{1,t-1}$ is the
same as $b_{2,t-2}$.

Assume that we have the initial conditions
$b_{10}=b_{20}=0$. The LHS of \eqref{eq:tsbud} describes the expenditures of the
agent, taking prices as given. The RHS represents period $t$
wealth. Given this, the Bellman equation is
\begin{align}
\label{eq:tsbel}
V(a_t,b_{1,t-1},b_{2,t-1};s_t)=\max_{a_{t+1},b_{1t},b_{2t}}\left\{u(c_t)+\beta\E[V(a_{t+1},b_{1t},b_{2t};s_{t+1})|s_t]\right\}\\
\text{s. t. }c_{t}+p_{t}a_{t+1}+q_{1t}b_{1t}+q_{2t}b_{2t}=(p_{t}+s_{t})a_{t}+b_{1,t-1}+q_{1t}b_{2,t-1}\nonumber
\end{align}
subject to all the appropriate non-negativity constraints. The relevant
market clearing conditions are $c_t=s_t$, $a_{t+1}=1$, $b_{1t}=b_{2t}=0$.


\item We can use the usual procedure of taking FOCs and envelope conditions
on the Bellman equation \eqref{eq:tsbel} and plugging in market clearing conditions
to find that:
\begin{eqnarray*}
p_t & = & \E_t\left[\beta\frac{u'(s_{t+1})}{u'(s_t)}(p_{t+1}+s_{t+1})\right]\\
1/R_{1t}=q_{1t} & = & \E_t\left[\beta\frac{u'(s_{t+1})}{u'(s_{t})}\right]\\
1/R_{2t}=q_{2t} & = & \E_t\left[\beta\frac{u'(s_{t+1})}{u'(s_t)}q_{1,t+1}\right]=\E_t\left[\beta^{2}\frac{u'(s_{t+2})}{u'(s_t)}\right]
\end{eqnarray*}

The last part of this question asks us to evaluate whether a reasonable
replication result holds. But as we will see, the combination of risk
aversion and uncertainty in this model will actually make this equation
not hold. In order to see this, we can compute:
\begin{eqnarray*}
q_{2t} & = & \mathbb{E}_{t}\left[\beta\frac{u'(s_{t+1})}{u'(s_{t})}q_{1,t+1}\right] \\
 & = & \mathbb{E}_{t}\left[\beta\frac{u'(s_{t+1})}{u'(s_{t})}\right]\mathbb{E}_{t}\left[q_{1,t+1}\right]+Cov_{t}\left(\beta\frac{u'(s_{t+1})}{u'(s_{t})},q_{1,t+1}\right) \\
 & = & q_{1t}\mathbb{E}_{t}\left[q_{1,t+1}\right]+Cov_{t}\left(\beta\frac{u'(s_{t+1})}{u'(s_{t})},q_{1,t+1}\right)\ne q_{1t}\mathbb{E}_{t}\left[q_{1,t+1}\right]\text{ in general. }
\end{eqnarray*}
Note that under risk neutrality, i.e. linear utility, marginal utility
is constant and so the covariance term goes away and the originally
proposed result holds.


\item For a zero-coupon bond with purchase price $1/R_{jt}$, the yield
to maturity is simply $YTM_{jt}=(R_{jt})^{1/j}$. Recalling that we
are now assuming that the dividends are i.i.d. to ease the exposition,
we can write the two yields as:
\begin{eqnarray*}
YTM_{1t} & = & \frac{1}{\beta}\left(\frac{u'(s_{t})}{\mathbb{E}(u'(s))}\right)\\
YTM_{2t} & = & \frac{1}{\beta}\left(\frac{u'(s_{t})}{\mathbb{E}(u'(s))}\right)^{1/2}
\end{eqnarray*}
Therefore, it follows that
\begin{equation*}
\frac{YTM_{2t}}{YTM_{1t}}=\left(\frac{\mathbb{E}(u'(s))}{u'(s_{t})}\right)^{1/2}.
\end{equation*}
Then, in order for the yield curve to be increasing we must have that
$\mathbb{E}(u'(s))>u'(s_{t})$. When consumption is high, marginal
utility is low, while expected future consumption is lower. So
everybody wants to save more in order to smooth their
consumption. Increased demand for bonds increases their price and
reduces interest rates, return to savings falls on both 1 and 2 year
bonds. But next period consumption is expected to revert to it's mean,
and interest rates are expected to be higher. Saving in 1 year bonds
can be reinvested then for a higher return, so demand for 1 year bonds
today is higher than for 2 year bonds, reducing their return more (in
annualized terms, or YTM).


\item Using the same approach that we used in parts 2 and 3, we can find
that the $j$-period ahead risk-free rate is given by
\begin{equation*}
1/R_{jt}=\mathbb{E}_{t}\left[\beta^{j}\frac{u'(s_{t+j})}{u'(s_{t})}\right],
\end{equation*}
and so the corresponding yield to maturity is given by
\begin{equation*}
YTM_{jt}=\frac{1}{\beta}\left[\frac{u'(s_{t})}{E(u'(s))}\right]^{1/j}
\end{equation*}
By iterating forward on the above expression, we can solve for the
relationships between yields of all maturities. Then, in general,
for $k>j$, 
\begin{equation*}
\frac{YTM_{kt}}{YTM_{jt}}=\left(\frac{\mathbb{E}(u'(s))}{u'(s_{t})}\right)^{(k-j)/kj},
\end{equation*}
and so the criteria for an increasing yield curve are the same as
above.

\end{enumerate}

}


\newpage

\section[]{Burrowed are?\footnote{Spring 2016 problem set}}

An economy consists of two types of consumens indexed by
$i=1,2$. There is one nonstorable consumption good. Let
$(e_t^i,c_t^i)$ be the endowment, consumption pair for consumer $i$ in
period $t$. Both consumers have preferences ordered by

\begin{equation*}
  U^i=\E\sum_{t=0}^\infty\beta^t\log(c_t^i)
\end{equation*}

The endowment streams of the two consumers are governed by two
independent 2-state Markov chains $s_t\in\{0,1\}$ with transition
matrix $P_s$ and $a_t\in\{0,1\}$ with transition matrix $P_a$, where
$P_{s,12}=Pr\{s_{t+1}=1|s_t=0\}$ and so on. Suppose that each chain is
stationary and that the initial states $(s_0,a_0)$ are drawn from the
stationary distribution, so that $Pr\{s_0=0\}=\overline{\pi}_s$ and
$Pr\{a_0=0\}=\overline{\pi}_a$. Consumer 1 has endowment
$e_t^1=a_t+s_t$, while consumer 2 has endowment $e_t^2=a_t+1-s_t$.

\begin{enumerate}[(a)]
\item Write out explicitly the preferences $U^i$ of a consumer in
  terms of the history of the states and their associated
  probabilities.
\item Define a competitive equilibrium for this economy.
\item Characterize the competitive equilibrium for this economy,
  calculating the prices of all Arrow-Debreu securities. How does the
  allocation vary across the $(s_t,a_t)$ states? Across time? Across
  consumers?
\item Now consider an economy with sequential trading in Arrow
  securities, one-period ahread claims to contingent consumption. How
  many Arrow securities are there? Compute their prices in the special
  case $\beta=0.95$, $P_{s,11}=0.9$, $P_{s,22}=0.8$, $P_{a,11}=0.8$,
  $P_{a,22}=0.7$.
\item Using these same parameters, in each state what is the price of
  a one-period ahead riskless claim to one unit of consumption?

\end{enumerate}

\sol{

  You can check details in the solution of Problem set 1, but here is
  a general overview of the algoritm.

  Solution relies on a few fundamental results, proven in LS Chapter 8.
  \begin{itemize}
  \item For a particular choice of prices date-0 trade (Arrow-Debreu)
    market equilibrium allocation is the same as in secquential
    (Arrow) market equilibrium.
  \item AD equilibrium is Pareto efficient.
  \item Solution to a social planner's problem is Pareto efficient.
  \end{itemize}

  \emph{Negishi algorithm} uses these results to solve for competitive
  equilibrium:
  \begin{enumerate}
  \item Solve social planner's problem with a set of Pareto
    weights. Solution is an Pareto efficient consumption sequence that
    depends on chosen weights.
  \item We know that \emph{one} of all possible Pareto efficient
    allocations coincides with AD market equilibrium. To select such
    allocation, we need to pick particular Pareto weights which will
    satisfy a feature of markets, budget constraint, and be consistent
    with household optimization. In addition, one price can be
    normalized to 1 (typically, price at time 0 and state 0. Once we
    found such weights, we can compute AD equilibrium.
  \item Sequential markets equilibrium has the same allocation as AD
    equilibrium and prices $q_{t+1}^t=q_{t+1}^0 / q_t^0$.
  \end{enumerate}

}

\end{document}
