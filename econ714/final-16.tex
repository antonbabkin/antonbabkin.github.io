\documentclass[12pt]{article}
\usepackage{amsfonts,amsmath}

\newtheorem{theorem}{Theorem}
\newtheorem{proposition}[theorem]{Proposition}

% PAGE DIMENSIONS
\setlength{\textwidth}{16.25cm} \setlength{\topmargin}{-1cm}
\setlength{\evensidemargin}{5cm} \setlength{\oddsidemargin}{0cm}
 \setlength{\textheight}{22.25cm}

\begin{document}
\begin{tabular}{lcr}
Noah Williams &\hspace*{150pt} & Economics 712 \\
Department of Economics && Macroeconomic Theory \\
University of Wisconsin && Spring 2016
\end{tabular}
\vspace{12pt}

\begin{center}
\textbf{\large Final Examination} \vspace{12pt}
\end{center}

\noindent \textbf{Instructions:}\ This is a 75 minute exam with
worth a total of 100 points. Point values on each part are marked.
\textbf{Allocate your time wisely.} In
order to get full credit, you must give a clear, concise, and
correct answer, including all necessary explanations and
calculations. Notes, books, and calculators are not permitted.

\bigskip

\begin{enumerate} 
\item \textbf{[50 points]} Consider a representative agent exchange economy with money, where
the aggregate endowment $Y_t$ is governed by an exogenous process:
\begin{equation}\label{output}
\log \frac{Y_{t}}{Y_{t-1}} = \mu  + \sigma W_{t}
\end{equation}
where $\mu \ge 0$ is the mean growth rate, and $W_t$ is an i.i.d.\ standard normal endowment
shock. Preferences over consumption $c_t$ and real money
balances $m_t= M_t/P_t$ are:
\[E_0 \sum_{t=0}^\infty \beta^t \left[\frac{c_t^{1-\gamma}}{1-\gamma} + v(m_t) \right],\]
where $v$ is strictly increasing, strictly concave,
and differentiable.  The agent can trade in  a
stock (claim to the endowment stream) with price $S_t$, a risk-free real bond (paying one unit of real goods) with 
price $1/R_t$, and a risk-free nominal bond (paying one unit of nominal goods with real value $P_{t}/P_{t+1}$)  with price $1/I_t$. 

Denote household wealth $x_t$ and suppose the agent is endowed with the stock and the initial
money: $x_0 = S_0 + M_0/P_0$. The agent then
chooses his consumption $c_t$, real money holdings $m_t$, holdings of
the real bond $\alpha_{bt}$, the nominal bond $\alpha_{B_t}$ and the
stock $\alpha_{St}$. The agent's wealth is then:
\[x_t =
\alpha_{bt} + \alpha_{Bt} + \alpha_{St} + m_t\] which satisfies
the budget constraint:
\begin{eqnarray*}
 x_{t+1} &=& x_t - c_t + \alpha_{b_t} (R_t-1) + \alpha_{Bt} (I_t-1) +
 \alpha_{St} r_t^s + \frac{M_t^s}{P_t} - m_t
\end{eqnarray*}\unskip
where $r_t^s=(Y_t + S_t)/S_{t-1}-1$ is the return on the stock.

\emph{Postmortem: This budget constraint appeared on the exam, but it
  is incorrect. See solution for details.}

\begin{enumerate}
\item \textbf{[20 points]} Find the agent's optimality conditions, then impose the equilibrium conditions (with nominal and real bonds in zero net supply) to characterize equilibrium prices and interest rates.
\item \textbf{[10 points]} Given the specification for the endowment process, solve explicitly for the net real interest rate $r_t=\log(R_t)$ and describe how it depends on the growth and volatility of output and the agent's preferences.
\item \textbf{[5 points]} We will solve for equilibria of the form $P_t = Y_t^a$ for some $a$. Define $\pi_t = \log E_t (P_{t+1}/P_t)$ as the net expected inflation rate.  Show that a given $\pi_t$ is (typically) consistent with two values of $a$.
\item \textbf{[5 points]} Solve for equilibrium nominal interest rate $i_t = \log(I_t)$ in this class of equilibria.  
\item \textbf{[10 points]} Suppose that monetary policy pegs a constant interest rate $i_t = \bar i$.  Show that if $\sigma = 0$ there is a unique equilibrium, but if $\sigma>0$ there are two equilibria.  Interpret your answer in terms of the Fisher equation and inflation risk.
\end{enumerate}


\item  \textbf{[30 points]} Consider a continuous time search model with variable intensity.  That is unemployed workers earn benefits $z$ and choose an intensity level $s$ which has (monetary) utility costs $c(s)$ which are increasing and convex, but increases the likelihood of finding a job $q(s)$ where $q$ is increasing and concave.  When employed, a worker earns a constant wage $w$ and the job is subject to destruction at rate $\lambda$.
\begin{enumerate}
\item \textbf{[20 points]} Write down the Hamilton-Jacobi-Bellman equations determining the value $U(s)$ of an unemployed worker who searches with intensity $s$ and $W$ of an employed worker.  Find the steady state values of $U(s)$ and $W$.
\item  \textbf{[5 points]} Characterize the optimal choice of $s$ for a currently unemployed worker (taking as given his search intensities in future unemployment spells), assuming $q'(s)=q(s)/s$.
\item  \textbf{[5 points]} How does the optimal search intensity in a steady state respond to an increase in the wage $w$?
\end{enumerate}

\item \textbf{[20 points]} Answer the following:
\begin{enumerate}
\item  \textbf{[7 points]}  Why is inflation costly in the New Keynesian model?
\item   \textbf{[7 points]} What is the equity premium puzzle and why can't it be resolved by higher $\gamma$ with preferences $u(c)=c^{1-\gamma}/(1-\gamma)$?
\item  \textbf{[6 points]} What is a time consistency problem and how does it arise in Ramsey optimal taxation?
\end{enumerate}



\end{enumerate}







\end{document}


