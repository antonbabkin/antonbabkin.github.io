\documentclass{article}
\usepackage{amsmath}
\usepackage{amssymb}
\usepackage{graphicx}
\usepackage{color}
\usepackage{enumerate}
\usepackage{etoolbox}
\usepackage[margin=1.5in]{geometry}

\newcommand{\R}{\mathbb{R}}
\newcommand{\E}{\mathbb{E}}

\title{Macro prelim solutions - August 2013}
\author{Anton Babkin}

\begin{document}

{\Large Macro prelim solutions - August 2013\footnote{By Anton
    Babkin. This version: \today.}}

\textit{Disclaimer: These are unofficial solutions, they might have
  errors and be incomplete. Your comments and corrections are welcome.}

\section*{Question 1.A.}

\textit{I am getting the opposite result in this problem: risk-free
  interest rate rises over time.}

\begin{enumerate}[(a)]
\item Recursive formulation of agent's problem:
  \begin{gather*}
    V(a_t,A_t,b_t)=\max \log(c_t) + \beta\E_t V(a_{t+1},A_{t+1},b_{t+1})\\
    \text{s.t. } c_t+p_ta_{t+1}+P_tA_{t+1}+q_tb_{t+1}=(p_t+d_t)a_t+(P_t+D_t)A_t+b_t
  \end{gather*}
  where asset $a_t$ has price $p_t$ and pays stochastic dividend
  $d_t$, asset $A_t$ has price $P_t$ and pays constant dividend
  $D_t=d_0$, and $b_t$ is a risk-free bond with price $q_t$.

  Recursive competitive equilibrium is sequence of quantities
  $c_t,a_t,A_t,b_t$ and prices $p_t,P_t,q_t$ such that:
  \begin{itemize}
  \item $c_t,a_t,A_t,b_t$ solve agent's problem taking prices as given,
  \item markets clear: $a_t=1$, $A_t=1$, $b_t=0$ and $c_t=d_t+D_t$.
  \end{itemize}

  \textit{It is not necessary to introduce risk-free bond into the
    model, risk-free rate can be computed without it using pricing
    kernel.}


\item Standard asset pricing Euler equations:
  \begin{gather*}
    p_t=\beta\E_t\frac{c_t}{c_{t+1}}(p_{t+1}+d_{t+1})\\
    P_t=\beta\E_t\frac{c_t}{c_{t+1}}(P_{t+1}+D_{t+1})\\
    q_t=\beta\E_t\frac{c_t}{c_{t+1}}\\
  \end{gather*}

  With market clearing conditions, risk-free bond price is
  \begin{equation*}
    q_t=\beta\E_t\frac{d_t+D_t}{d_{t+1}+D_{t+1}}=\beta\E_t\frac{d_t+d_0}{d_{t+1}+d_0}\\
  \end{equation*}

  Using law of motion for $d_t$,
  $E_t(f(d_{t+1}))=\pi f(d_t(1+\delta))+(1-\pi)f(d_t)$. Then
  \begin{align*}
    q_t&=\beta[\pi\frac{d_t+d_0}{d_{t}(1+\delta)+d_0}+(1-\pi)\frac{d_t+d_0}{d_{t}+d_0}]\\
       &=\beta[\pi\frac{d_t+d_0}{d_{t}(1+\delta)+d_0}+(1-\pi)]\\
       &=\beta[\pi\frac{d_t(1+\delta)+d_0-d_t\delta}{d_{t}(1+\delta)+d_0}+(1-\pi)]\\
       &=\beta[1-\pi\frac{d_t\delta}{d_{t}(1+\delta)+d_0}]\\
  \end{align*}

  In expectation as of $t=0$, future change in price is
  \begin{align*}
    \E_0[q_{t+1}-q_t]&=\E_0\left[\E_t[q_{t+1}-q_t]\right]\\
                     &=\E_0\left[
                       \E_t\left(
                       \beta[1-\pi\frac{d_{t+1}\delta}{d_{t+1}(1+\delta)+d_0}]
                       \right)-
                       \beta[1-\pi\frac{d_t\delta}{d_{t}(1+\delta)+d_0}]
                       \right]\\
                     &=\E_0\left[\beta\pi\left(\frac{d_t\delta}{d_{t}(1+\delta)+d_0}-\E_t[\frac{d_{t+1}\delta}{d_{t+1}(1+\delta)+d_0}]\right)\right]\\
                     &=\E_0\left[\beta\pi\left(\frac{d_t\delta}{d_{t}(1+\delta)+d_0}-\pi\frac{d_{t}\delta(1+\delta)}{d_{t}(1+\delta)^2+d_0}-(1-\pi)\frac{d_{t}\delta}{d_{t}(1+\delta)+d_0}\right)\right]\\
                     &=\E_0\left[\beta\pi^2\left(\frac{d_{t}\delta}{d_{t}(1+\delta)+d_0}-\frac{d_{t}\delta(1+\delta)}{d_{t}(1+\delta)^2+d_0}\right)\right]\\
                     &=\E_0\left[\beta\pi^2\left(\frac{d_{t}\delta(1+\delta)}{d_{t}(1+\delta)^2+d_0(1+\delta)}-\frac{d_{t}\delta(1+\delta)}{d_{t}(1+\delta)^2+d_0}\right)\right]
  \end{align*}
  
  \begin{align*}
  d_{t}(1+\delta)^2+d_0(1+\delta) &> d_{t}(1+\delta)^2+d_0\\
  \frac{d_{t}\delta(1+\delta)}{d_{t}(1+\delta)^2+d_0(1+\delta)}&<\frac{d_{t}\delta(1+\delta)}{d_{t}(1+\delta)^2+d_0}\\
  \E_0[q_{t+1}-q_t]&<0
  \end{align*}

  $\E_0q_{t+1}<\E_0q_t$, risk-free bond price $q_t$ falls over time in
  expectation, so risk-free interest rate increases over
  time. Intuitively, asset $a$ is becoming more and more valuable over
  time as it pays increasing dividend, so return on an alternative
  asset such as risk-free bond $b$ must be also increasing so that
  it's market clears in equilibrium.

\item Using the law of iterated expectations,
  \begin{align*}
    \E_0d_t&=\E_0\E_{t-1}d_t\\
           &=\E_0d_{t-1}(\pi(1+\delta)+(1-\pi))\\
           &=\E_0\E_{t-2}d_{t-1}(1+\pi\delta)\\
           &=\E_0d_{t-2}(1+\pi\delta)^2\\
           &...\\
           &=d_0(1+\pi\delta)^t
  \end{align*}

  So $\lim_{t\to\infty}\E_0d_t=\infty$, in expectation dividend grows
  without bound.

  Then the limit of the bond price is:
  \begin{align*}
    \lim_{t\to\infty}\E_0q_t&=\lim_{t\to\infty}\E_0\beta[1-\pi\frac{d_t\delta}{d_{t}(1+\delta)+d_0}]\\
                            &=\lim_{t\to\infty}\E_0\beta[1-\pi\frac{\delta}{(1+\delta)+d_0/d_{t}}]\\
                            &=\beta[1-\pi\frac{\delta}{1+\delta}]<\beta
  \end{align*}

  So in the limit the risk-free rate is $>1/\beta$.


\item Iterating Euler equation forward and assuming no-bubble equilibrium, obtain

  \begin{align*}
    \frac{p_t}{d_t}&=\E_t\sum_{j=1}^\infty\beta^j\frac{d_t+d_0}{d_{t+j}+d_0}\frac{d_{t+j}}{d_t}\\
                   &=\frac{d_t+d_0}{d_t}\E_t\sum_{j=1}^\infty\beta^j\frac{d_{t+j}}{d_{t+j}+d_0}
  \end{align*}

  As $t$ increases, $\frac{d_t+d_0}{d_t}$ decreases at a decelerating
  rate. So the term $\frac{d_t+d_0}{d_t}$ decreases faster than each
  of the $\frac{d_{t+j}}{d_{t+j}+d_0}$ terms is increasing. Then price
  to dividend ratio is declining. In the limit
  $d_{t+j}\approx d_{t+j}+d_0$, so
  \begin{equation*}
    \lim_{t\to\infty}\E_0\frac{p_t}{d_t}=\sum_{j=1}^\infty\beta^j=\frac{\beta}{1-\beta}
  \end{equation*}

\end{enumerate}


\end{document}




























