\documentclass{article}

\usepackage{amsmath}
\usepackage{color}
\usepackage{graphicx}
\usepackage[margin=1.5in]{geometry}

\begin{document}

\textbf{Macro Example Comp Question from Quarter 1, 2015}:

\textit{With solution}

\bigskip

Analyze the dynamic equilibrium of a two period lived overlapping
generations production economy with a constant unit measure of identical
agents in each generation. Each agent's preferences are given by the utility
function 
\[
u(c_{t+1}^{t},n_{t}^{t})=\frac{1}{1-\gamma }(c_{t+1}^{t})^{1-\gamma
}-n_{t}^{t} 
\]%
where $c_{t+1}^{t}$ denotes period $t+1$ consumption of an agent born in
period $t$ (i.e. an old person), $n_{t}^{t}\in \lbrack 0,1]$ denotes labor
supplied in period $t$ by an agent born in period $t$ (i.e. a young person).
and $\gamma \in (0,1)$. Technology is given by the production function $%
f(k_{t},n_{t})=k_{t}^{1/2}n_{t}^{1/2}$ where $k_{t}$ is total capital
available to each producer in period $t.$ Capital fully depreciates after
production, so $k_{t+1}=i_{t}$ where $i_{t}$ is investment chosen by the
young in period $t$ from their labor earnings. Young agents born in period $%
t $ supply labor $n_{t}^{t}$ at real wage $w_{t}$ in order to buy capital $%
k_{t+1}$ which they rent to firms in their second period of life at real
gross return $R_{t+1}$to obtain funds $R_{t+1}k_{t+1}$for purchasing
consumption goods $c_{t+1}^{t}$.

\begin{enumerate}
\item Write down the optimization problem faced by a generation $t$ agent.
Solve for labor supply and investment decision rules. (5 points)

\textit{Answer:}

Capital depreciates fully, so $k_{t+1}=i_t$, and we can write that
households choose capital directly. Firm ownership does not matter as
technology is CRS, and profits will be zero in equilibrium.

  \begin{gather*}
    \max_{c_{t+1}^t,n_t^t,
      k_{t+1}}\frac{1}{1-\gamma}(c_{t+1}^{t})^{1-\gamma}-n_{t}^{t}\\
    \text{s.t. } k_{t+1}=w_tn_t^t\\
    c_{t+1}^t=R_{t+1}k_{t+1}\\
    c_{t+1}^t, k_{t+1}\ge 0\\
    n_{t}^t\in [0,1]
  \end{gather*}

Assume that parameters are such that $n_t^t\in(0,1)$, i.e. inequality
constraints are not binding. Utility function satisfies Inada
conditions, so $c_{t+1}^t\ge 0$ is not binding either.

Using $c_{t+1}=R_{t+1}k_{t+1}=R_{t+1}w_tn_t^t$, rewrite household
problem as
\begin{equation*}
  \max_{n_t^t}\frac{1}{1-\gamma}(R_{t+1}w_tn_t^t)^{1-\gamma}-n_{t}^{t}
\end{equation*}

Take FOC in $n_t^t$:
\begin{equation*}
  R_{t+1}^{1-\gamma}w_t^{1-\gamma}(n_t^t)^{-\gamma}=1
\end{equation*}

Solve for labor supply, and plug into budget constraint for investment:
\begin{gather*}
  n_t^t=R_{t+1}^\frac{1-\gamma}{\gamma}w_t^\frac{1-\gamma}{\gamma}\\
  k_{t+1}=R_{t+1}^\frac{1-\gamma}{\gamma}w_t^\frac{1}{\gamma}
\end{gather*}


\item Write down the optimization problem faced by a representative firm
which rents labor at price $w_{t}$ and capital at gross rate $R_{t}$ to
maximize real profits. (2.5 points)

\textit{Answer:}

Firm's maximizes profit taking prices $R_t,w_t$ as given:

\begin{gather*}
  \max_{K_t,N_t} y_t - R_tK_t-w_tN_t\\
\text{s.t. } y_t = K_{t}^{1/2}N_{t}^{1/2}
\end{gather*}

Labor and capital demand desicions satisfy FOCs:
\begin{eqnarray*}
  R_t=\frac{1}{2}K_t^{-\frac{1}{2}}N_t^{\frac{1}{2}}\\
  w_t=\frac{1}{2}K_t^{\frac{1}{2}}N_t^{-\frac{1}{2}}
\end{eqnarray*}

\item Define a competitive equilibrium. (2.5 points)

\textit{Answer:}

Competitive equilibrium is an allocation $i_t,k_t,n_t,y_t,c_{t+1}^t$ and prices $R_t,w_t$ such that $\forall t$:
\begin{itemize}
\item Allocation solves household and firm optimization problems.
\item Markets clear:
  \begin{itemize}
  \item labor: $n_t=N_t$,
  \item capital: $k_t=K_t$,
  \item goods: $c_t^{t-1}+i_t=y_t$
  \end{itemize}
\end{itemize}

\item Show that a competitive equilibrium satisfies the following pair of
first order difference equations%
\begin{eqnarray*}
k_{t+1} &=&R_{t}k_{t}, \\
R_{t+1}^{1-\gamma } &=&4k_{t}^{\gamma }R_{t}^{1+\gamma }.
\end{eqnarray*}%
(10 points)

\textit{Answer:}



Goods market clearing:
\begin{eqnarray*}
  k_{t+1}+c_t^{t-1}=k_t^{\frac{1}{2}}n_t^{\frac{1}{2}}
\end{eqnarray*}

Using HH budget constraints and firm FOCs:
\begin{eqnarray*}
  k_{t+1}+R_tk_t=k_t^{\frac{1}{2}}2R_tk_t^{\frac{1}{2}}
\end{eqnarray*}

This gives the first difference equation:
\begin{equation*}
  k_{t+1}=R_tk_t
\end{equation*}

Use firm FOCs to eliminate $n$ and $w$ from HH FOCs:
\begin{eqnarray*}
  R_{t+1}^{1-\gamma}(\frac{1}{4R_t})^{1-\gamma}(4R_t^2k_t)^{-\gamma}=1
\end{eqnarray*}

Simplify to get the second difference equation:
\begin{equation*}
  R_{t+1}^{1-\gamma }=4k_{t}^{\gamma }R_{t}^{1+\gamma }
\end{equation*}


\item Describe stationary competitive equilibria in $(k_{t},R_{t})$ space
and investigate their stability. (12.5 points)

\textit{Answer:}

Dynamics of the model is characterised by a two-dimensional system of
first order difference equations:

\begin{eqnarray*}
  k_{t+1}&=&R_tk_t\\
  R_{t+1}&=&4^\frac{1}{{1-\gamma}}k_{t}^\frac{\gamma}{1-\gamma}R_{t}^\frac{1+\gamma}{1-\gamma}
\end{eqnarray*}

or

\begin{equation*}
  \begin{pmatrix}
    k_{t+1}\\
    R_{t+1}
  \end{pmatrix}
= g(  \begin{pmatrix}
    k_{t}\\
    R_{t}
  \end{pmatrix})
\end{equation*}

In steady state $R$ and $k$ solve
\begin{gather*}
  k=Rk\\
  R^{1-\gamma }=4k^{\gamma}R^{1+\gamma}
\end{gather*}

There are two solutions: $(k,R)=(0,0)$ and $(k,R)=(4^{-\frac{1}{\gamma}}, 1)$.

To characterize stability of steady states we need matrix of first derivatives
\begin{equation*}
  Dg(k_t,R_t)=
  \begin{bmatrix}
    g_{11} & g_{12}\\
    g_{21}&g_{22}
  \end{bmatrix}=
  \begin{bmatrix}
    R_t & k_t\\
    4^\frac{1}{{1-\gamma}}k_{t}^\frac{\gamma}{1-\gamma}R_{t}^\frac{1+\gamma}{1-\gamma} \frac{\gamma}{1-\gamma}k_t^{-1} &
    4^\frac{1}{{1-\gamma}}k_{t}^\frac{\gamma}{1-\gamma}R_{t}^\frac{1+\gamma}{1-\gamma} \frac{1+\gamma}{1-\gamma}R_t^{-1}
  \end{bmatrix}
\end{equation*}

This is not well defined in $(0,0)$, so we can't characterize
stability of the trivial steady state. We will focus on the second
steady state, $(k,R)=(4^{-\frac{1}{\gamma}}, 1)$.

\begin{equation*}
  Dg(k,R)=
  \begin{bmatrix}
    1 & 4^{-\frac{1}{\gamma}}\\
    \frac{\gamma}{1-\gamma}4^{\frac{1}{\gamma}}&\frac{1+\gamma}{1-\gamma}
  \end{bmatrix}
\end{equation*}

Eigenvalues $\lambda$ of this matrix solve characteristic equation
$p(\lambda)\equiv\lambda^2-T\lambda+D=0$, where
\begin{eqnarray*}
  T&=&g_{11}+g_{22}=\frac{2}{1-\gamma}\\
  D&=&g_{11}g_{22}-g_{21}g_{12}=\frac{1}{1-\gamma}
\end{eqnarray*}

Discriminant $\mathcal{D}=T^2-4D=\frac{4\gamma}{(1-\gamma)^2}>0$, so
  there are 2 different real eigenvalues.

  \begin{equation*}
    \lambda=\frac{\frac{2}{1-\gamma}\pm\sqrt{\frac{4\gamma}{(1-\gamma)^2}}}{2}=
    \frac{1\pm\sqrt{\gamma}}{1-\gamma}=\frac{1}{1\pm\sqrt{\gamma}}
  \end{equation*}


  So the two roots of the equation are such that $0<\lambda_1=\frac{1}{1+\sqrt{\gamma}}<1$ and
  $\lambda_2=\frac{1}{1-\sqrt{\gamma}}>1$. Steady state is stable, and the system converges
  along a saddle path.

\item Suppose that the economy starts in a non-trivial steady
  state. Using a phase diagram, describe the dynamics of
  $(k_{t},R_{t})$ after an unexpected negative shock to the capital
  stock (i.e. part of the capital accumulated by the young cohort is
  destroyed when they become old). Further, what happens to wages,
  labor supply, and consumption over time in response to the shock?
  (17.5 points)

\textit{Answer:}

\begin{eqnarray*}
  k_{t+1}&=&R_tk_t\\
  R_{t+1}&=&4^\frac{1}{{1-\gamma}}k_{t}^\frac{\gamma}{1-\gamma}R_{t}^\frac{1+\gamma}{1-\gamma}
\end{eqnarray*}

\begin{equation*}
k=const:\quad R_t=1\\
\end{equation*}

$k_t$ increases if $R_t>1$ and decreases if $R_t<1$.

\begin{equation*}
R=const:\quad R=4^{-\frac{1}{2\gamma}}k^{-\frac{1}{2}}
\end{equation*}

Rewrite the second difference equation as
\begin{equation*}
  \left(\frac{R_{t+1}}{R_t}\right)^{1+\gamma}=4k_t^\gamma R_{t+1}^{2\gamma}
\end{equation*}

Above the $R=const$ line $k$ is large and $R$ is large, so
$\frac{R_{t+1}}{R_t}>0$ and $R_t$ increases, below the line -
decreases.

\begin{figure}[h]
  \begin{center}
    \includegraphics[width=0.7\textwidth]{phase}
  \end{center}
  \caption{Phase diagram.}
\end{figure}

If capital drops below steady state level, $R$ will immediatelly jump
up to the saddle path. Over time, $k$ will grow and $R$ will decline
back to the steady state.

Other variables of the model can be expressed as functions of the
state vector $(k_t,R_t)$:

\begin{align*}
  n_t&=4R_t^2k_t\\
  w_t&=\frac{1}{4R_t}\\
  y_t&=k_{t}^{1/2}n_{t}^{1/2}=2R_tk_t\\
  c_t^{t-1}&=R_tk_t
\end{align*}

The first two equations follow from firm's FOCs, and the last one from
household budget constraint. Clearly, on impact wage $w_0$ will
fall. Change in other variables depends on relative changes in $k$ and
$R$. Effects can be either computed using numerical methods, or
evaluated with log-linear approximation around steady state.

First-order Taylor approximation of the system of difference equations is
\begin{equation*}
  \begin{pmatrix}
    k_{t+1}-k\\
    R_{t+1}-R
  \end{pmatrix}
= Dg(k,R)  \begin{pmatrix}
    k_{t}-k\\
    R_{t}-R
  \end{pmatrix}
\end{equation*}

Saddle path can be found as an eigenvector $\mathbf{v}$ that
corresponds to eigenvalue $0<\lambda_1<1$. $\mathbf{v}$ solves
$(Dg(k,R)-I\lambda_1)\mathbf{v}=0$. Normalize first element of
$\mathbf{v}$ to $v_1=1$, so that $\mathbf{v}=(1,v_2)$.

\begin{equation*}
    \begin{bmatrix}
    1-\frac{1}{1+\sqrt{\gamma}}& 4^{-\frac{1}{\gamma}}\\
    \frac{\gamma}{1-\gamma}4^{\frac{1}{\gamma}}&\frac{1+\gamma}{1-\gamma}-\frac{1}{1+\sqrt{\gamma}}
  \end{bmatrix}
  \begin{pmatrix}
    1\\
    v_2
  \end{pmatrix}=
\begin{pmatrix}
    0\\
    0
  \end{pmatrix}
\end{equation*}

\begin{equation*}
  \begin{cases}
    \frac{\sqrt{\gamma}}{1+\sqrt{\gamma}}+4^{-\frac{1}{\gamma}}v_2=0\\
    \frac{\gamma}{1-\gamma}4^{\frac{1}{\gamma}}+\frac{\sqrt{\gamma}}{1-\sqrt{\gamma}}v_2=0
  \end{cases}
\end{equation*}

Two equations are identical, solving either one yields
$v_2=-\frac{\sqrt{\gamma}}{1+\sqrt{\gamma}}4^{\frac{1}{\gamma}}$.

$k_0-k$ is the initial shock to capital. Then $R_0$ must be on the
saddle path. With linear approximation, slope of the saddle path is
equal to the slope of the eigenvector, so
\begin{equation*}
  \frac{R_0-R}{k_0-k}=\frac{v_2}{v_1}=-\frac{\sqrt{\gamma}}{1+\sqrt{\gamma}}4^{\frac{1}{\gamma}}  
\end{equation*}

Let hats denote deviations from steady state, e.g.
$\hat{R}_t=\frac{R_t-R}{R}$. Then

\begin{equation*}
  \hat{R}_0=-\frac{\sqrt{\gamma}}{1+\sqrt{\gamma}}\hat{k}_0
\end{equation*}

Log-linearize the four equations for other variables of the model:
\begin{align*}
  \hat{n}_t&=2\hat{R}_t+\hat{k}_t\\
  \hat{w}_t&=-\hat{R}_t\\
  \hat{y}_t&=\hat{R}_t+\hat{k}_t\\
  \hat{c}_t^{t-1}&=\hat{R}_t+\hat{k}_t
\end{align*}

Plug in expression for $\hat{R}_0$, and with negative shock
$\hat{k}_0<0$:
\begin{align*}
  \hat{n}_0&=\frac{1-\sqrt{\gamma}}{1+\sqrt{\gamma}}\hat{k}_0<0\\
  \hat{w}_0&=\frac{\sqrt{\gamma}}{1+\sqrt{\gamma}}\hat{k}_0<0\\
  \hat{y}_0&=\frac{1}{1+\sqrt{\gamma}}\hat{k}_0<0\\
  \hat{c}_0^{-1}&=\frac{1}{1+\sqrt{\gamma}}\hat{k}_0<0
\end{align*}

Labor, wages, output and consumption all decrease below their steady
state levels in response to a negative shock to capital, and then
gradually converge back.

\end{enumerate}

\end{document}
