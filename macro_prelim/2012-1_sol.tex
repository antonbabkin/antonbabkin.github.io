\documentclass{article}
\usepackage{amsmath}
\usepackage{amssymb}
\usepackage{graphicx}
\usepackage{color}
\usepackage{enumerate}
\usepackage{etoolbox}
\usepackage[margin=1.5in]{geometry}

\newcommand{\R}{\mathbb{R}}
\newcommand{\E}{\mathbb{E}}

\title{Macro prelim solutions - June 2012}
\author{Anton Babkin}

\begin{document}

{\Large Macro prelim solutions - June 2012\footnote{By Anton
    Babkin. This version: \today.}}

\textit{Disclaimer: These are unofficial solutions, they might have
  errors and be incomplete. Your comments and corrections are welcome.}

\
\section*{Question 2A}

Looks like there is a typo in the utility function, and I change it to
the following:

\begin{equation*}
  u(t,c_t)=\left(\Pi_{s=1}^t\beta_s\right)\log(c_t)
\end{equation*}

\textit{I also assume that in the recursive problem, discount factor
  for the next period is unknown when decision is made,
  $V(\beta)=\max u+\E\beta'V(\beta')$. Another, possibly even better,
  interpretation could be $V(\beta)=\max u+\beta\E V(\beta')$.}

\begin{enumerate}[(a)]
\item Recursive problem:
  \begin{gather*}
    V(d_t,\beta_t)=\max \log(c_t)+\E\beta_{t+1}V(d_{t+1},\beta_{t+1})\\
    \text{s.t. } c_t+P_ta_{t+1}+q_tb_{t+1}=(P_t+d_t)a_t+b_t
  \end{gather*}
Standard asset pricing Euler equations with  market clearing condition $c_t=d_t$:
  \begin{gather*}
    P_t=\E_t\beta_{t+1}\frac{d_t}{d_{t+1}}(P_{t+1}+d_{t+1})\\
    q_t=\E_t\beta_{t+1}\frac{d_t}{d_{t+1}}\\
  \end{gather*}

  With $\beta$ and $d$ following Markov process, state at $t$ has
  information about state at $t+1$. For example, if process is
  persistent and $\beta_t$ is high, then $\E_t\beta_{t+1}$ is also
  high. Then then agents are more patient, and prices of both assets
  will be higher, risk-free rate $R=1/q$ and expected return on the tree -
  lower. When $d_t$ is high, it will go down in expectation, so
  $\E_t\frac{d_t}{d_{t+1}}$ is higher, so the prices are higher.

  \item 
    \begin{align*}
      q_t&=\E_t\beta_{t+1}\frac{d_t}{d_{t+1}}\\
         &=d_t\left(\pi\frac{\beta_1}{d_1}+(1-\pi)\frac{\beta_2}{p_2}\right)
    \end{align*}

    Risk-free rate:
    \begin{equation*}
      R_t = \frac{1}{q_t}=\frac{1}{d_t\left(\pi\frac{\beta_1}{d_1}+(1-\pi)\frac{\beta_2}{p_2}\right)}
    \end{equation*}

  \item Risk-free rate with constant $\beta$ will be lower if
    \begin{equation*}
      \beta(\frac{\pi}{d_1}+\frac{1-\pi}{d_2})>\pi\frac{\beta_1}{d_1}+(1-\pi)\frac{\beta_2}{p_2}
    \end{equation*}
    Simplifying this inequality yields
    \begin{equation*}
      \beta_2>\beta_1
    \end{equation*}

\end{enumerate}



\section*{Question 3}
  Parts of this problem can be interpreted differently. I assume the following:
  \begin{itemize}
  \item Productivity shock $\varepsilon$ is drawn independently for
    every period, cohort and household. I.e. old parent can have
    different productivity from when he was young.
  \item Intra-vivos transfer $i$ is chosen by parents when they are
    old, in the third period.
  \item Borrowing is not allowed.
  \end{itemize}

\begin{enumerate}
 \item Young parent:
  \begin{gather*}
    V(a,i,h,\varepsilon,a_k,b)=\max_{n,c_y,e,e_k,s}u(c_y)+\beta\E_{a'|a,\varepsilon'}J(s,b,a_k,a',h_o',h_k',\varepsilon')\\
    \text{s.t. } c_y+e+e_k+s=i+wh(1-n)\varepsilon\\
    h_o'=a(nh)^{\gamma_1}e^{\gamma_2}+(1-\delta)h\\
    h_k'=a_kh^{\gamma_1}e_k^{\gamma_2}+(1-\delta)h
  \end{gather*}

Old parent:
\begin{gather*}
  J(s,b,a_k,a',h_o',h_k',\varepsilon')=\max_{c_o',i',b'}u(c_o')+\theta V(a_k,i',h_k',\varepsilon',a',b')\\
  \text{s.t. }c_o'+i'+b'=Rb+Rs+wh_o'\varepsilon'  
\end{gather*}

\item \textit{Plug in $J$ into $V$, and use standard argument of
    Blackwell's sufficient condition (monotonicity and discounting)
    for $V$.}

\item \textit{Plug $J$ into $V$ and apply Stokey-Lucas theorems.}

\item Plug in $c_y,h_o',h_k',c_o$ from constraints.

FOC:
\begin{eqnarray*}
  n:&wh\varepsilon u'(c_y)=\beta\E J_5(t)\frac{dh_o'}{dn}\\
  e:&u'(c_y)=\beta\E J_5(t)\frac{dh_o'}{de}\\
  e_k:&u'(c_y)=\beta\E J_6(t)\frac{dh_k'}{de_k}\\
  s:&u'(c_y)=\beta\E J_1(t)\\
  i':&u'(c_o')=\theta V_2(t+1)\\
  b':&u'(c_o')=\theta V_6(t+1)
\end{eqnarray*}
where
\begin{eqnarray*}
  \frac{dh_o'}{dn}=\gamma_1an^{\gamma_1-1}h^{\gamma_1}e^{\gamma_2}\\
  \frac{dh_o'}{dn}=\gamma_2a(nh)^{\gamma_1}e^{\gamma_2-1}\\
  \frac{dh_k'}{de_k}=\gamma_2a_k(h)^{\gamma_1}e_k^{\gamma_2-1}
\end{eqnarray*}

ENV:
\begin{eqnarray*}
  J_5(t)=u'(c_o')w\varepsilon'\\
  J_6(t)=\theta V_3(t+1)\\
  J_1(t)=u'(c_o')R\\
  V_2(t)=u'(c_y)\\
  V_6(t)=\beta\E J_2(t)\\
  J_2(t)=Ru'(c_o')\\
  V_3(t)=u'(c_y)w(1-n)\varepsilon+\beta\E[J_5(t)\frac{dh_o'}{dh}+J_6(t)\frac{dh_k'}{dh}]
\end{eqnarray*}
where
\begin{eqnarray*}
  \frac{dh_o'}{dh}=\gamma_1an^{\gamma_1}h^{\gamma_1-1}e^{\gamma_2}\\
  \frac{dh_k'}{dh}=\gamma_1a_k(h)^{\gamma_1-1}e_k^{\gamma_2}
\end{eqnarray*}

Euler equations:
\begin{eqnarray*}
  n:&wh\varepsilon u'(c_y)=\beta\E u'(c_o')w\varepsilon'\frac{dh_o'}{dn}\\
  e:&u'(c_y)=\beta\E u'(c_o')w\varepsilon'\frac{dh_o'}{de}\\
  e_k:&u'(c_y)=\beta\E \frac{dh_k'}{de_k}\theta\left[
       u'(c_y')w(1-n')\varepsilon'
       +\beta u'(c_o'')w\varepsilon''\frac{dh_o''}{dh'}
       +\beta \frac{dh_k''}{dh'}\theta V_3(t+2)
       \right]=...\\
  s:&u'(c_y)=\beta \E Ru'(c_o')\\
  b':&u'(c_o')=\theta\beta\E Ru'(c_o'')\\
  i':&u'(c_o')=\theta u'(c_y')
\end{eqnarray*}

Interpretations:
\begin{itemize}
\item $n$: Marginal benefit of working and hence earning more when
  young must be equal to marginal benefit of having higher human
  capital and hence earning more when old.
\item $e$: Marginal cost of investing in own human capital and
  giving up consumption when young must be equal to marginal benefit
  of having higher human capital and hence earning more when old.
\item $e_k$: Marginal cost of investing in child's human capital and
  giving up consumption when young must be equal to it's marginal
  benefit that comes in three ways: children earning more when they
  are young adults, children having higher human capital when they get
  old and hence earning more, and their children having higher human
  capital. The last term recursively unfolds into the infinite future,
  so optimal investment into children accounts for utilities of the
  whole dynasty.
\item $s$: Marginal utility of consuming when young equals marginal utility
  of consuming more when old with added return on savings.
\item $b'$: Marginal utility of consumtion when old equals marginal
  utility of children next period when they become old, adjusted for
  how much parents care about children, $\theta$.
\item $i'$: Marginal utility of old parents equals marginal utility of
  their children, adjusted by $\theta$.
\end{itemize}

\item The difference, as discussed above, is that investment in
  children has far-reaching effect on all subsequent generations,
  whereas investment in own human capital only gives benefit in the
  next period.

\item There are two types of incompleteness in this
  environment. First, borrowing constraint: young parents can't borrow
  ($s\ge 0$, and old parents can't pass debt onto their heirs
  ($b\ge 0$). Second, there are no insurance markets,
  i.e. state-contingent claims. The first causes over-accumulation of
  physical and human capital due to a precautionary motive. Absence of
  contingent markets makes inter-temporal decisions (savings,
  investment in children, bequests) efficient ex ante (in
  expectation), but inefficient ex post. Under complete markets,
  savings will be lower. Investment into children will be such that
  their human capital only depends on their ability, but not on their
  parents' wealth. Allocation of resoures will be socially optimal.

\item Because the tax is proportional and the transfer is lump sum,
  such policy will result in redistribution from the rich (high
  productivity, high human capital, hence high labor earnings) to the
  poor (low productivity, low human capital). Redistribution effect
  will make the rich oppose such intervention.

  On the other hand, such fiscal policy provides a form of insurance
  and consumption smoothing over different realizations of
  productivity and ability of future generations. This insurance
  channel will be valuable for the rich.

\item In the standard Aiyagari-Bewley model agents are
  over-accumulating physical capital to protect themselves from a
  sequence of bad shocks in the future, because there is a potentially
  binding borrowing constraint. If technology exhibits diminishing
  product of capital, interest rate is inversely related to
  capital. Capital is above socially optimal level, so interest rate
  is below socially optimal (which is equal to the rate of time
  preference).

  In the model with human capital, this might not be true. As in
  Aiyagari-Bewley, market incompleteness makes agents over-accumulate
  capital, both physical and human. When technology uses both types of
  capital, interest rate depends negatively on physical capital, but
  positively on human capital. Depending on model parameters, either
  effect might dominate, so interest rates may be above or below the
  rate of time preference.

\item Notice that such redistribution was already available in the
  form of intra-vivos transfers. Now a certain amount $T$ of such
  transfer becomes mandatory. It will not affect househods who's
  transfer was already big, $i>T$, they will merely reduce $i$ by the
  amount of $T$. But other households - relatively poor parents of
  relatively rich children - who choose $i<T$, will be worse off,
  because they can't reduce their transfer $i$ below zero.

\end{enumerate}




\end{document}




























